
        \documentclass[twoside=true,titlepage=false,open=any, parskip=never, fontsize=10pt, headings=small, chapterprefix=false, appendixprefix=false]{scrbook}
        \addtolength{\oddsidemargin}{\evensidemargin}
        \setlength{\oddsidemargin}{.5\oddsidemargin}
        \setlength{\evensidemargin}{\oddsidemargin}
        \usepackage[paperheight=230mm, paperwidth=138mm, textwidth=100mm, textheight=185mm]{geometry}
        
        \usepackage{scrlayer-scrpage}
        \usepackage{hyphenat}
        \usepackage{fontspec}
        \usepackage{moresize}
        %\usepackage{ipa}
        \usepackage[english, ngerman]{babel}
        \usepackage[babel]{microtype}
        \usepackage{soul}
        \usepackage{scrhack}
        \usepackage{xpatch}
        \usepackage[nonewpage]{indextools}
        \usepackage{lettrine}
        \usepackage[series={A,B,C}]{reledmac}
        
        \setmainfont[Path=./fonts/,
        Extension=.ttf,
        UprightFont=*-Roman,
        BoldFont=*-Bold,
        ItalicFont=*-Italic,
        BoldItalicFont=*-BoldItalic]{Brill}
        
        \KOMAoptions{toc=chapterentrydotfill, toc=flat}
        \addtokomafont{chapterentrypagenumber}{\mdseries}
        \setkomafont{chapterentry}{\normalfont\mdseries}
        \setkomafont{partentry}{\normalfont\mdseries}
        \RedeclareSectionCommand[tocbeforeskip=0pt]{chapter}         
        
        \setlength{\skip\footins}{4mm plus 2mm} % Abstand Fussnote Text
        \interfootnotelinepenalty=10000 % Kein Seitenwechsel in Fuss
        
        %% Sperrung (Package Soul)
        %% Hier ist die Sperrung definiert. Sperrung erreicht man mit \so{gesperrtes Wort}
        \sodef\so{}{.14em}{.4em plus.1em minus .1em}{.4em plus.1em minus .1em}
        
        
        
        % Fußnoten linksbündig
        \deffootnote{1.5em}{1em}{% 
        \makebox[1.5em][l]{\thefootnotemark}%
        }
        
        
        % Fussnotenlineal (wobei für reledmac wohl was anderes gilt)
        \let\normalfootnoterule\footnoterule
        \setfootnoterule{0pt}
        \let\normalfootnoterule\footnoterule
        
        
        \setlength{\skip\footins}{8mm plus 2mm} % Abstand Fussnote Text
        \interfootnotelinepenalty=10000 % Kein Seitenwechsel in Fuss
        
        %% Kapitelüberschriften
        \renewcommand*{\raggedchapter}{\centering} 
        \renewcommand*{\raggedsection}{%
        \CenteringLeftskip=1cm plus 1em\relax 
        \CenteringRightskip=1cm plus 1em\relax 
        \Centering\footnotesize\thesection{}.\ }
        \setkomafont{section}{\footnotesize}
        \setkomafont{chapter}{\normalfont\Large}
        \renewcommand{\chapterpagestyle}{empty}%The first page in each chapter won't have any heading or footer, especially no page number
        
        % section ohne führende Kapitelnummer
        \renewcommand*\thesection{\arabic{section}}
        
        % Bildunterschrift ohne Nummer
        \renewcommand*{\figureformat}{}
        \renewcommand*{\captionformat}{}
        
        %% Zeilennummern
        \firstlinenum{0} \linenumincrement{5}
        \lineation{section} %Jeder Abschnitt wird durchnummeriert
        \renewcommand{\numlabfont}{\ssmall} %Schriftgröße Zeilennummern
        
        
        \newcommand{\Theight}{\dimexpr\fontcharht\font`W}
        \newcommand{\pbposition}{\depth}
        \newcommand{\pb}{\nobreak\hspace{0pt}\raisebox{-0.1em}{\raisebox{\pbposition}{\textnormal{HANSI4EVER}}}\nobreak\hspace{0pt}}
        
        \renewcaptionname{ngerman}{\contentsname}{Inhalt}           %Table of contents
        
        % FUSSNOTE
        %% Im Apparat f. und ff.
        \Xtwolines{f.}
        \Xtwolinesbutnotmore
        
        %% Zeilennummerierung Abstand zum Lemma
        \Xboxlinenum{5mm}
        
        %% Bei zwei Apparateinträgen in einer Zeile wird nur beim ersten Mal die Zeile gezählt
        \Xnumberonlyfirstinline
        \Xnumberonlyfirstintwolines
        \Xinplaceofnumber{1em}
        \Xhangindent{1em}
        
        % ENDNOTEN
        \Xendbeforepagenumber{} 
        \Xendparagraph[A]
        \Xendsep{}
        \Xendafterpagenumber[A]{, }
        \Xendnotenumfont[A]{\tiny}
        \Xendlineprefixsingle[A]{\tiny}
        \Xendlineprefixmore[A]{\tiny}
        \Xendhangindent[A]{4em}
        
        \Xendboxlinenum[A]{0em}
        \Xendlemmaseparator{$\rbracket$}
        \Xendnotefontsize[A]{\footnotesize}
        \Xendhangindent[A]{0em}
        \Xendlemmafont[A]{\normalfont}
        \Xendlemmafont[B]{\bfseries}
        \Xendnotefontsize[B]{\footnotesize}
        \Xendnotenumfont{\footnotesize}
        \Xendlineprefixsingle[C]{\tiny}
        \Xendlineprefixmore[C]{\tiny}
        \Xendlemmadisablefontselection
        \Xendlemmafont{\itshape}
        \Xendlinerangeseparator{\tiny{--}}
        \Xendhangindent{4em}
        \Xendboxlinenum{3.6em}
        \Xendafternumber{0.4em}
        \Xendboxlinenumalign{R}
        
        \makeindex[name=kw,title=Schlagworteregister,columns=2]
        \makeindex[name=person,title=Personenregister,columns=2]
        
        \begin{document}

        \beginnumbering
        
            
            
        \pstart
        ÜBER DEN TRAUM
        \pend
    
         
            
            
            
        \pstart
        „Über den Traum“ erschien zuerst 1901 im
                  Verlag J. F. Bergmann, Wiesbaden (seither München), in den „Grenzfragen des
                  Nerven- und Seelenlebens“ (herausg. von Löwenfeld und Kurella); 2. Auflage 1911,
                  3. Auflage 1922. Übersetzungen erschienen in folgenden Sprachen: englisch (von M.
                  D. Eder), London 1914; russisch, Petersburg 1909; holländisch (von J. Stärcke),
                  Leiden 1913; ungarisch (von S. Ferenczi), Budapest 1915; italienisch (von M.
                  Levi-Bianchini), 1919; dänisch (von Gelsted, zusammen mit der Übersetzung der fünf
                  Vorlesungen „Über Psychoanalyse“), 1920; spanisch (von Luis Lopez Ballesteros y de
                  Torres in Bd. II der Obras Completas), Madrid 1923, und französisch (von Helene
                  Legros in „Les Documents Bleus“), Paris 1925.
        \pend
    
         
            
            
            \pstart\eledsection*{
               \edtext{I}{\lemma{\textbf{I}}\Aendnote{I. E  D1D2 }}}\pend
            
        \pstart
        In \edtext{den}{\lemma{\textbf{den}}\Aendnote{ D2D1E }} Zeiten, die wir vorwissenschaftliche nennen dürfen, waren die Menschen um
               die Erklärung des Traumes\edindex[kw]{Schlagwort Nr. 2748} nicht verlegen.
               Wenn sie ihn nach dem Erwachen erinnerten, galt er ihnen als eine entweder
               gnädige oder feindselige Kundgebung höherer, dämonischer und göttlicher\edtext{}{\lemma{\textbf{}}\Aendnote{, ED1D2 }} Mächte. Mit dem Aufblühen naturwissenschaftlicher Denkweisen
               hat sich all diese sinnreiche Mythologie\edindex[kw]{Schlagwort Nr. 1830} in Psychologie umgesetzt, und heute
               bezweifelt nur mehr eine geringe Minderzahl unter den Gebildeten, \edtext{daß}{\lemma{\textbf{daß}}\Aendnote{dass ED1D2 }} der Traum die eigene psychische Leistung
               des Träumers ist.
        \pend
    
            
        \pstart
        Seit der Verwerfung der mythologischen\edindex[kw]{Schlagwort Nr. 1830} Hypothese ist der
               Traum\edindex[kw]{Schlagwort Nr. 2748} aber erklärungsbedürftig geworden. Die Bedingungen
               seiner Entstehung, seine Beziehung zum Seelenleben des Wachens\edindex[kw]{Schlagwort Nr. 3169}, seine Abhängigkeit von Reizen, die sich während des Schlafzustandes\edindex[kw]{Schlagwort Nr. 2337} zur Wahrnehmung\edindex[kw]{Schlagwort Nr. 3186} drängen, die vielen dem wachen Denken \edtext{anstößigen}{\lemma{\textbf{anstößigen}}\Aendnote{anstössigen ED1D2 }} Eigentümlichkeiten seines Inhaltes, die Inkongruenz zwischen seinen
               Vorstellungsbildern und den an sie geknüpften
               Affekten\edindex[kw]{Schlagwort Nr. 52}, endlich die Flüchtigkeit des Traumes, die Art,
               wie das wache Denken ihn als fremdartig \edtext{beiseite}{\lemma{\textbf{beiseite}}\Aendnote{bei Seite ED1D2 beiseiteschiebt S }}
               \edtext{schiebt}{\lemma{\textbf{schiebt}}\Aendnote{ S }}, in der Erinnerung\edindex[kw]{Schlagwort Nr. 685} verstümmelt oder auslöscht:
               — all diese und noch andere Probleme verlangen seit vielen hundert
               Jahren nach Lösungen, die bis heute nicht befriedigend gegeben werden konnten.
               Im Vordergrunde des Interesses steht aber die Frage nach der Bedeutung des Traumes, die einen zweifachen Sinn
               in sich \edtext{schließt}{\lemma{\textbf{schließt}}\Aendnote{schliesst ED1D2 }}. Sie
        \pend
    
         
            
            
            
        \pstart
        fragt erstens nach der psychischen Bedeutung des Träumens, nach der
               Stellung des Traumes zu anderen seelischen Vorgängen\edindex[kw]{Schlagwort Nr. 2432}
               und nach einer etwaigen biologischen Funktion desselben, und zweitens
               möchte sie wissen, ob der Traum deutbar ist, ob der einzelne
               Trauminhalt\edindex[kw]{Schlagwort Nr. 2766} einen „Sinn“
               hat, wie wir ihn in anderen psychischen Kompositionen zu finden gewöhnt
               sind.
        \pend
    
            
        \pstart
        Drei Richtungen machen sich in der Würdigung des Traumes\edindex[kw]{Schlagwort Nr. 2748} bemerkbar. Die eine derselben, die gleichsam den Nachklang
               der alten \edtext{Überschätzung}{\lemma{\textbf{Überschätzung}}\Aendnote{Ueberschätzung E }} des Traumes bewahrt hat, findet ihren Ausdruck bei manchen
               Philosophen. Ihnen gilt als die Grundlage des Traumlebens ein besonderer Zustand
               der \edtext{Seelentätigkeit}{\lemma{\textbf{Seelentätigkeit}}\Aendnote{Seelenthätigkeit E }}, den sie sogar als eine Erhebung zu einer höheren Stufe feiern. So \edtext{urteilt}{\lemma{\textbf{urteilt}}\Aendnote{urtheilt E }}
                z. B. Schubert\edindex[person]{Person Nr. 4}:
               Der Traum sei eine Befreiung des Geistes von der Gewalt der \edtext{äusseren}{\lemma{\textbf{äusseren}}\Aendnote{äußeren CS }} Natur, eine Loslösung der Seele von den Fesseln der Sinnlichkeit. Andere
               Denker gehen nicht so weit, halten aber daran fest, \edtext{daß}{\lemma{\textbf{daß}}\Aendnote{dass ED1D2 }} die Träume wesentlich seelischen Anregungen entspringen und \edtext{Äußerungen}{\lemma{\textbf{Äußerungen}}\Aendnote{Aeusserungen E Äusserungen D1D2 }} seelischer Kräfte darstellen, die tagsüber an ihrer freien
               Entfaltung behindert sind (der
               Traumphantasie\edindex[kw]{Schlagwort Nr. 2770} — Scherner\edindex[person]{Person Nr. 3}, Volkelt\edindex[person]{Person Nr. 2}). Eine Fähigkeit zur
               \edtext{Überleistung}{\lemma{\textbf{Überleistung}}\Aendnote{Ueberleistung E }} wenigstens auf gewissen Gebieten (\edtext{Gedächtnis\edindex[kw]{Schlagwort Nr. 3413}}{\lemma{\textbf{Gedächtnis\edindex[kw]{Schlagwort Nr. 3413}}}\Aendnote{Gedächtniss E }}) wird dem Traumleben von einer \edtext{großen}{\lemma{\textbf{großen}}\Aendnote{grossen ED1D2 }} Anzahl von Beobachtern zugesprochen.
        \pend
    
            
        \pstart
        Im scharfen Gegensatz \edtext{hiezu}{\lemma{\textbf{hiezu}}\Aendnote{hierzu ED1D2 }} vertritt die Mehrzahl ärztlicher Autoren eine Auffassung, welche dem Traum\edindex[kw]{Schlagwort Nr. 2748} kaum noch den \edtext{Wert}{\lemma{\textbf{Wert}}\Aendnote{Werth E }}
                eines psychischen Phänomens \edtext{beläßt}{\lemma{\textbf{beläßt}}\Aendnote{belässt ED1D2 }}. Die Erreger des Traumes\edindex[kw]{Schlagwort Nr. 3412} sind nach ihnen \edtext{ausschließlich}{\lemma{\textbf{ausschließlich}}\Aendnote{ausschliesslich ED1D2 }} die Sinnes- und Leibreize\edindex[kw]{Schlagwort Nr. 2221}, die entweder von \edtext{außen}{\lemma{\textbf{außen}}\Aendnote{aussen ED1D2 }} den Schläfer treffen oder zufällig in seinen inneren Organen rege werden.
               Das Geträumte hat nicht mehr Anspruch auf Sinn und Bedeutung als etwa die
               Tonfolge, welche die zehn Finger eines der Musik\edindex[kw]{Schlagwort Nr. 1806}
               ganz unkundigen Menschen hervorrufen, wenn sie über die Tasten des \edtext{Instruments}{\lemma{\textbf{Instruments}}\Aendnote{Instrumentes ED1D2 }} hinlaufen. Der Traum ist geradezu als „ein körperlicher, in allen Fällen
                  unnützer, in vielen Fällen krankhafter Vorgang“ zu kenn-
        \pend
    
         
            
            
            
        \pstart
        zeichnen (Binz\edindex[person]{Person Nr. 1}). Alle \edtext{Eigentümlichkeiten}{\lemma{\textbf{Eigentümlichkeiten}}\Aendnote{Eigenthümlichkeiten E }} des Traumlebens erklären sich aus der \edtext{zusammenhanglosen}{\lemma{\textbf{zusammenhanglosen}}\Aendnote{zusammenhangslosen ED1D2 }}, durch physiologische Reize erzwungenen Arbeit einzelner Organe oder
               Zellgruppen des sonst in Schlaf versenkten Gehirns.
        \pend
    
            
        \pstart
        Wenig \edtext{beeinflußt}{\lemma{\textbf{beeinflußt}}\Aendnote{beeinflusst ED1D2 }} durch dieses \edtext{Urteil}{\lemma{\textbf{Urteil}}\Aendnote{Urtheil E }} der Wissenschaft und unbekümmert um die Quellen des Traumes \edtext{,}{\lemma{\textbf{,}}\Aendnote{ D1ED2 }} scheint die Volksmeinung an dem Glauben festzuhalten, \edtext{daß}{\lemma{\textbf{daß}}\Aendnote{dass ED1D2 }} der Traum denn doch einen Sinn habe, der sich auf die Verkündigung der
               Zukunft bezieht \edtext{,}{\lemma{\textbf{,}}\Aendnote{ S }} und der durch \edtext{irgend ein}{\lemma{\textbf{irgend ein}}\Aendnote{irgendein S }} Verfahren der Deutung aus seinem oft verworrenen und \edtext{rätselhaften}{\lemma{\textbf{rätselhaften}}\Aendnote{räthselhaften E }} Inhalt gewonnen werden könne. Die in Anwendung gebrachten
               Deutungsmethoden bestehen darin, \edtext{daß}{\lemma{\textbf{daß}}\Aendnote{dass ED1D2 }} man den erinnerten Trauminhalt durch einen anderen ersetzt, entweder
               Stück für Stück nach einem feststehenden
                  Schlüssel, oder das Ganze des Traumes durch ein anderes Ganzes, zu dem
               es in der Beziehung eines Symbols steht.
               Ernsthafte Männer lächeln über diese Bemühungen. „Träume sind Schäume.“
        \pend
    
         
            
            
            \pstart\eledsection*{II}\pend
            
        \pstart
        Zu meiner großen Überraschung entdeckte ich eines Tages, daß nicht die
               ärztliche, sondern die laienhafte, halb noch im Aberglauben befangene
               Auffassung des Traumes der Wahrheit nahe kommt. Ich gelangte nämlich zu neuen
               Aufschlüssen über den Traum, indem ich eine neue Methode der psychologischen
                  Untersuchung auf ihn anwendete, die mir bei der Lösung der
               Phobien, Zwangsideen, Wahnideen u. dgl. hervorragend gute Dienste geleistet
               hatte, und die seither unter dem Namen „Psychoanalyse“ bei einer ganzen Schule
               von Forschern Aufnahme gefunden hat. Die mannigfaltigen Analogien des
               Traumlebens mit den verschiedenartigsten Zuständen psychischer
               Krankheit im Wachen sind ja von zahlreichen ärztlichen Forschern mit Recht
               bemerkt worden. Es erschien also von vorneherein hoffnungsvoll, ein
               Untersuchungsverfahren, welches sich bei den psychopathischen Gebilden bewährt
               hatte, auch zur Aufklärung des Traumes heranzuziehen. Die Angst- und
               Zwangsideen stehen dem normalen Bewußtsein in ähnlicher Weise fremd gegenüber
               wie die Träume dem Wachbewußtsein; ihre Herkunft ist dem Bewußtsein ebenso
               unbekannt wie die der Träume. Bei diesen psychopathischen Bildungen wurde man
               durch ein praktisches Interesse getrieben, ihre Herkunft und Entstehungsweise zu
               ergründen, denn die Erfahrung hatte gezeigt, daß eine solche Aufdeckung der
               dem Bewußtsein verhüllten Gedankenwege, durch welche die krankhaften Ideen mit dem übrigen psychischen Inhalt zusammen-
        \pend
    
         
            
            
            
        \pstart
        hängen, einer Lösung dieser Symptome gleichkommt, die Bewältigung der bisher unhemmbaren Idee zur Folge hat. Aus der Psychotherapie
               stammte also das Verfahren, dessen ich mich für die Auflösung der Träume
               bediente.
        \pend
    
            
        \pstart
        Dieses Verfahren ist leicht zu beschreiben, wenngleich seine Ausführung
               Unterweisung und Übung erfordern dürfte. Wenn man es bei einem anderen, etwa
               einem Kranken mit einer Angstvorstellung, in Anwendung zu bringen
               hat, so fordert man ihn auf, seine Aufmerksamkeit auf die betreffende Idee zu
               richten, aber nicht, wie er schon so oft getan, über sie nachzudenken,
               sondern alles ohne Ausnahme sich klar zu machen und
               dem Arzt mitzuteilen, was ihm zu ihr einfällt.
               Die dann etwa auftretende Behauptung, daß die Aufmerksamkeit nichts
               erfassen könne, schiebt man durch eine energische Versicherung, ein solches
               Ausbleiben eines Vorstellungsinhaltes sei ganz unmöglich, zur Seite. Tatsächlich
               ergeben sich sehr bald zahlreiche Einfälle, an die sich weitere knüpfen, die
               aber regelmäßig von dem Urteil des Selbstbeobachters eingeleitet werden, sie
               seien unsinnig oder unwichtig, gehören nicht hierher, seien ihm nur zufällig
               und außer Zusammenhang mit dem gegebenen Thema eingefallen. Man merkt
               sofort, daß es diese Kritik ist, welche all diese
               Einfälle von der Mitteilung, ja bereits vom Bewußtwerden, ausgeschlossen
               hat. Kann man die betreffende Person dazu bewegen, auf solche Kritik gegen ihre
               Einfälle zu verzichten und die Gedankenreihen, die sich bei festgehaltener
               Aufmerksamkeit ergeben, weiter zu spinnen, so gewinnt man ein psychisches
               Material, welches alsbald deutlich an die zum Thema genommene krankhafte Idee
               anknüpft, deren Verknüpfungen mit anderen Ideen bloßlegt, und in weiterer
               Verfolgung gestattet, die krankhafte Idee durch eine neue zu ersetzen, die sich
               in verständlicher Weise in den seelischen Zusammenhang einfügt.
        \pend
    
            
        \pstart
        Es ist hier nicht der Ort, die Voraussetzungen, auf denen dieser Versuch ruht,
               und die Folgerungen, die sich aus seinem regel-
        \pend
    
         
            
            
            
        \pstart
        mäßigen Gelingen ableiten, ausführlich zu behandeln. Es mag also die
               Aussage genügen, daß wir bei jeder krankhaften Idee ein zur Lösung derselben
               hinreichendes Material erhalten, wenn wir unsere Aufmerksamkeit gerade den „ungewollten“, den „unser Nachdenken störenden“, den sonst von der Kritik als wertloser
               Abfall beseitigten Assoziationen zuwenden. Übt man das Verfahren an sich selbst,
               so unterstützt man sich bei der Untersuchung am besten durch sofortiges
               Niederschreiben seiner anfänglich unverständlichen Einfälle.
        \pend
    
            
        \pstart
        Ich will nun zeigen, wohin es führt, wenn ich diese Methode der Untersuchung auf
               den Traum anwende. Es müßte jedes Traumbeispiel sich in gleicher Weise dazu
               eignen; aus gewissen Motiven wähle ich aber einen eigenen Traum, der mir in
               der Erinnerung undeutlich und sinnlos erscheint, und der sich durch seine
               Kürze empfehlen kann. Vielleicht wird gerade der Traum der letzten Nacht diesen
               Ansprüchen genügen. Sein unmittelbar nach dem Erwachen fixierter Inhalt lautet
               folgendermaßen:
        \pend
    
            
        \pstart
        „Eine Gesellschaft, Tisch oder Table
                     d’hôte\edtext{}{\Bendnote{\textbf{Table d’hôte}
                     Gasthaustafel mit festgelegtem Menü, im Gegensatz zu à la carte. Vgl. SFs Brief
                     an Martha Freud, Riva, 5.9.1900: „u im nahe gelegenen [Hotel] Mahlzeiten an der
                     Tabel des Todes, wie ich sie reichlicher u besser, leichter u appetitlicher
                     nirgends getroffen habe.“ (Reisebriefe, S.130).}}
     . . . Es wird
                  Spinat gegessen . . . Frau E. L. sitzt neben mir, wendet sich ganz mir
                  zu und legt vertraulich die Hand auf mein Knie. Ich entferne die Hand
                  abwehrend. Sie sagt dann: Sie haben aber immer so schöne Augen gehabt . . .
                  Ich sehe dann undeutlich etwas wie zwei Augen als Zeichnung oder wie die
                  Kontur eines Brillen glases . . .“
        \pend
    
            
        \pstart
        Dies ist der ganze Traum oder wenigstens alles, was ich von ihm erinnere. Er
               erscheint mir dunkel und sinnlos, vor allem aber befremdlich. Frau E. L. ist
               eine Person, zu der ich kaum je freundschaftliche Beziehungen gepflogen, meines
               Wissens herzlichere nie gewünscht habe. Ich habe sie lange Zeit nicht
               gesehen und glaube nicht, daß in den letzten Tagen von ihr die Rede war.
               Irgendwelche Affekte haben den Traumvorgang nicht begleitet.
        \pend
    
            
        \pstart
        Nachdenken über diesen Traum bringt ihn meinem Verständnis
        \pend
    
         
            
            
            
        \pstart
        nicht näher. Ich werde aber jetzt absichts- und kritiklos die Einfälle verzeichnen, die sich meiner Selbstbeobachtung\edtext{}{\Bendnote{\textbf{meiner Selbstbeobachtung} Von Freud\edindex[person]{Person Nr. 5} zur Methode ausgebaut; steht in Kontrast zu jener häufig
                  anzutreffenden Selbstbeobachtung, die Unzusammenhängendes im Traum vorschnell als
                  unsinnig oder unwichtig verwirft.}}
     ergeben. Ich bemerke bald, daß es
               dabei vorteilhaft ist, den Traum in seine Elemente zu zerlegen und zu jedem
               dieser Bruchstücke die anknüpfenden Einfälle aufzusuchen.
        \pend
    
            
        \pstart
        Gesellschaft, Tisch oder Table d’hôte. Daran
               knüpft sich sofort die Erinnerung an das kleine Erlebnis, welches den
               gestrigen Abend beschloß. Ich war von einer kleinen Gesellschaft weggegangen in
               Begleitung eines Freundes, der sich erbot, einen Wagen zu nehmen und mich nach
               Hause zu führen. „Ich ziehe einen Wagen mit Taxameter vor,“ sagte er, „das
               beschäftigt einen so angenehm; man hat immer etwas, worauf man schauen
               kann.“ Als wir im Wagen saßen und der Kutscher die Scheibe einstellte, so daß
               die ersten sechzig Heller sichtbar wurden, setzte ich den Scherz fort. „Wir sind
               kaum eingestiegen und schulden ihm schon sechzig Heller. Mich erinnert der
               Taxameterwagen immer an die Table d’hôte. Er
               macht mich geizig und eigensüchtig, indem er mich unausgesetzt an
               meine Schuld mahnt. Es kommt mir vor, daß diese zu schnell wächst, und ich
               fürchte mich, zu kurz zu kommen, gerade wie ich mich auch an der
               Table d’hôte der komischen Besorgnis, ich bekomme zu
               wenig, müsse auf meinen Vorteil bedacht sein, nicht erwehren kann.“ In
               entfernterem Zusammenhange hiermit zitierte ich:
        \pend
    
            
        \pstart
        „Ihr führt ins Leben uns hinein,
                  Ihr laßt den Armen schuldig werden.\edtext{}{\Bendnote{
                     \textbf{„Ihr führt ins Leben uns hinein} Lied des Harfners aus Goethes
                     Wilhelm Meisters Lehrjahre, Goethe: Sämtl. Werke (Münchner Ausgabe), Bd. 5,
                        S. 134)}}
    “
        \pend
    
            
        \pstart
        Ein zweiter Einfall zur Table d’hôte: Vor einigen Wochen habe ich mich an einer
                  Gasthaustafel in einem Tiroler Höhenkurort
               heftig über meine liebe Frau geärgert, die mir nicht reserviert genug gegen
               einige Nachbarn war, mit denen ich durchaus keinen Verkehr anknüpfen
               wollte. Ich bat sie, sich mehr mit mir als mit den Fremden zu beschäftigen. Das
               ist ja auch, als ob ich an der Table d’hôte zu kurz
                  gekommen wäre. Jetzt fällt mir auch der Gegensatz auf zwischen dem
        \pend
    
         
            
            
            
        \pstart
        Benehmen meiner Frau an jener Tafel und dem der Frau E. L. im Traum,
                  „die sich ganz mir zuwendet“.
        \pend
    
            
        \pstart
        Weiter: Ich merke jetzt, daß der Traumvorgang die Reproduktion einer
               kleinen Szene ist, die sich ganz ähnlich so zwischen meiner Frau und mir zur
               Zeit meiner geheimen Werbung zugetragen hat. Die Liebkosung unter dem
               Tischtuch war die Antwort auf einen ernsthaft werbenden Brief. Im
               Traum ist aber meine Frau durch die mir fremde E. L. ersetzt.
        \pend
    
            
        \pstart
        Frau E. L. ist die Tochter eines Mannes, dem ich Geld
                  geschuldet habe! Ich kann nicht umhin zu bemerken, daß sich da ein
               ungeahnter Zusammenhang zwischen den Stücken des Trauminhalts und meinen
               Einfällen enthüllt. Folgt man der Assoziationskette, die von einem Element des
               Trauminhaltes ausgeht, so wird man bald zu einem anderen Element
               desselben zurückgeführt. Meine Einfälle zum Traume stellen Verbindungen
               her, die im Traume selbst nicht ersichtlich sind.
        \pend
    
            
        \pstart
        Pflegt man nicht, wenn jemand erwartet, daß andere für seinen Vorteil sorgen
               sollen, ohne eigenen Vorteil dabei zu finden, diesen Weltunkundigen höhnisch zu
               fragen: Glauben sie denn, daß dies oder jenes um
                  Ihrer schönen Augen willen geschehen wird? Dann bedeutet ja die Rede der
               Frau E. L. im Traume: „Sie haben immer so schöne Augen gehabt“ nichts anderes
               als: Ihnen haben die Leute immer alles zu Liebe getan; Sie haben alles umsonst gehabt. Das Gegenteil ist natürlich wahr:
               Ich habe alles, was mir andere etwa Gutes erwiesen, teuer bezahlt. Es muß
               mir doch einen Eindruck gemacht haben, daß ich gestern den Wagen umsonst gehabt habe, in dem mich mein Freund nach
               Hause geführt hat.
        \pend
    
            
        \pstart
        Allerdings der Freund, bei dem wir gestern zu Gaste waren, hat mich oft zu
               seinem Schuldner gemacht. Ich habe erst unlängst eine Gelegenheit, es ihm zu
               vergelten, ungenützt vorübergehen lassen. Er hat ein einziges Geschenk von mir,
               eine antike Schale, auf der ringsum Augen gemalt
               sind, ein sog. Occhiale zur
        \pend
    
         
            
            
            
        \pstart
        Abwehr des Malocchio. Er ist übrigens Augenarzt.
               Ich hatte ihn an demselben Abend nach der Patientin gefragt, die ich zur
                  Brillenbestimmung in seine Ordination
               empfohlen hatte.
        \pend
    
            
        \pstart
        Wie ich bemerke, sind nun fast sämtliche Stücke des Trauminhaltes in
               den neuen Zusammenhang gebracht. Ich könnte aber konsequenter Weise noch fragen,
               warum im Traume gerade Spinat aufgetischt wird? Weil Spinat an eine kleine Szene erinnert, die kürzlich an unserem
               Familientische vorfiel, als ein Kind — gerade jenes, dem man die schönen Augen wirklich nachrühmen kann
               — sich weigerte, Spinat zu essen. Ich selbst benahm mich als Kind ebenso; Spinat war mir lange Zeit ein Abscheu, bis sich
               mein Geschmack später änderte und dieses Gemüse zur Lieblingsspeise erhob. Die
               Erwähnung dieses Gerichts stellt so eine Annäherung her zwischen meiner Jugend
               und der meines Kindes. „Sei froh, daß du Spinat hast,“ hatte die Mutter dem
               kleinen Feinschmecker zugerufen. „Es gibt Kinder, die mit Spinat sehr zufrieden
               wären.“ Ich werde so an die Pflichten der Eltern gegen ihre Kinder erinnert. Die
               Goetheschen Worte:
        \pend
    
            
        \pstart
        „Ihr führt ins Leben uns hinein, Ihr laßt den Armen schuldig werden“
        \pend
    
            
            
        \pstart
        zeigen in diesem Zusammenhange einen neuen Sinn.
        \pend
    
            
        \pstart
        Ich werde hier haltmachen, um die bisherigen Ergebnisse der Traumanalyse zu
               überblicken. Indem ich den Assoziationen folgte, welche sich an die einzelnen,
               aus ihrem Zusammenhang gerissenen Elemente des Traumes anknüpften, bin ich zu
               einer Reihe von Gedanken und Erinnerungen gelangt, in denen ich wertvolle
               Äußerungen meines Seelenlebens erkennen muß. Dieses durch die Analyse des
               Traumes gefundene Material steht in einer innigen Beziehung zum Trauminhalt,
               doch ist diese Beziehung von der Art, daß ich das neu Gefundene niemals aus dem
               Trauminhalt hätte erschließen können. Der Traum war affektlos, unzusammen-
        \pend
    
         
            
            
            
        \pstart
        hängend und unverständlich; während ich die Gedanken hinter dem
               Traume entwickle, verspüre ich intensive und gut begründete Affektregungen; die
               Gedanken selbst fügen sich ausgezeichnet zu logisch verbundenen
               Ketten zusammen, in denen gewisse Vorstellungen als zentrale wiederholt
               vorkommen. Solche im Traum selbst nicht vertretene Vorstellungen sind in
               unserem Beispiel die Gegensätze von eigennützig—uneigennützig, die Elemente
                  schuldig sein und umsonst tun. Ich könnte in dem Gewebe, welches sich der Analyse
               enthüllt, die Fäden fester anziehen und würde dann zeigen können, daß sie zu
               einem einzigen Knoten zusammenlaufen, aber Rücksichten nicht wissenschaftlicher, sondern privater Natur hindern mich, diese Arbeit
               öffentlich zu tun. Ich müßte zu vielerlei verraten, was besser mein Geheimnis
               bleibt, nachdem ich auf dem Wege zu dieser Lösung mir allerlei klar gemacht, was
               ich mir selbst ungern eingestehe. Warum ich aber nicht lieber einen
               anderen Traum wählte, dessen Analyse sich zur Mitteilung besser eignet, so
               daß ich eine bessere Überzeugung für den Sinn und Zusammenhang des durch
               Analyse aufgefundenen Materials erwecken kann? Die Antwort lautet, weil jeder Traum, mit dem ich mich beschäftigen will,
               zu denselben schwer mitteilbaren Dingen führen und mich in die gleiche Nötigung
               zur Diskretion versetzen wird. Ebensowenig würde ich diese
               Schwierigkeit vermeiden, wenn ich den Traum eines anderen zur Analyse brächte,
               es sei denn, daß die Verhältnisse gestatteten, ohne Schaden für den mir
               Vertrauenden alle Verschleierungen fallen zu lassen.
        \pend
    
            
        \pstart
        Die Auffassung, die sich mir schon jetzt aufdrängt, geht dahin, daß der Traum
               eine Art Ersatz ist für jene affektvollen und
               sinnreichen Gedankengänge, zu denen ich nach vollendeter Analyse gelangt bin.
               Ich kenne den Prozeß noch nicht, welcher aus diesen Gedanken den Traum hat
               entstehen lassen, aber ich sehe ein, daß es Unrecht ist, diesen als einen rein
               körperlichen, psychisch bedeutungslosen Vorgang hinzustellen, der durch die
               isolierte
        \pend
    
         
            
            
            
        \pstart
        Tätigkeit einzelner, aus dem Schlaf geweckter Hirnzellgruppen
               entstanden ist.
        \pend
    
            
        \pstart
        Zweierlei merke ich noch an: daß der Trauminhalt sehr viel kürzer ist als die
               Gedanken, für deren Ersatz ich ihn erkläre, und daß die Analyse eine unwichtige
               Begebenheit des Abends vor dem Träumen als den Traumerreger aufgedeckt hat.
        \pend
    
            
        \pstart
        Ich werde einen so weit reichenden Schluß natürlich nicht ziehen, wenn mir erst
               eine einzige Traumanalyse vorliegt. Wenn mir aber die Erfahrung gezeigt hat, daß
               ich durch kritiklose Verfolgung der Assoziationen von jedem Traum aus zu einer solchen Kette von
               Gedanken gelangen kann, unter deren Elementen die Traumbestandteile
               wiederkehren, und die unter sich korrekt und sinnreich verknüpft sind, so wird
               die geringe Erwartung, daß die das erstemal bemerkten Zusammenhänge sich als
               Zufall herausstellen könnten, wohl aufgegeben werden. Ich halte mich
               dann für berechtigt, die neue Einsicht durch Namengebung zu fixieren. Den
               Traum, wie er mir in der Erinnerung vorliegt, stelle ich dem durch Analyse
               gefundenen zugehörigen Material gegenüber, nenne den ersteren den manifesten Trauminhalt, das letztere — zunächst
               ohne weitere Scheidung — den latenten Trauminhalt.
               Ich stehe dann vor zwei neuen, bisher nicht formulierten Problemen: 1) welches der psychische Vorgang ist, der den latenten
               Trauminhalt in den mir aus der Erinnerung bekannten, manifesten, übergeführt
               hat; 2) welches das Motiv oder die Motive sind, die
               solche Übersetzung erfordert haben. Den Vorgang der Verwandlung vom latenten zum
               manifesten Trauminhalt werde ich die Traumarbeit
               nennen. Das Gegenstück zu dieser Arbeit, welches die entgegengesetzte
               Umwandlung leistet, kenne ich bereits als Analysenarbeit. Die anderen Traumprobleme, die Fragen nach den
               Traumerregern, nach der Herkunft des Traummaterials, nach dem etwaigen Sinn
               des Traumes und Funktion des Träumens, und nach den Gründen des
               Traumvergessens werde ich nicht am manifesten, sondern am
        \pend
    
         
            
            
            
        \pstart
        neugewonnenen latenten Trauminhalt erörtern. Da ich alle
               widersprechenden wie alle unrichtigen Angaben über das Traumleben in
               der Literatur auf die Unkenntnis des erst durch Analyse zu enthüllenden latenten
               Trauminhaltes zurückführe, werde ich eine Verwechslung des manifesten Traumes mit den
               latenten Traumgedanken fortan aufs sorgfältigste zu
                  vermeiden suchen.
        \pend
    
         
            
            
            \pstart\eledsection*{III}\pend
            
        \pstart
        Die Verwandlung der latenten Traumgedanken in den mani festen Trauminhalt
               verdient unsere volle Aufmerksamkeit als das zuerst bekannt gewordene Beispiel
               von Umsetzung eines psychischen Materials aus der einen
               Ausdrucksweise in die andere, aus einer Ausdrucksweise, die uns ohne weiteres
               verständlich ist, in eine andere, zu deren Verständnis wir erst durch Anleitung
               und Bemühung vordringen können, obwohl auch sie als Leistung unserer \edtext{Seelentätigkeit}{\lemma{\textbf{Seelentätigkeit}}\Aendnote{Seelenthätigkeit E }} anerkannt werden muß. Mit Rücksicht auf das Verhältnis von latentem zu
               manifestem Trauminhalt lassen sich die Träume in drei Kategorien bringen. Wir
               können erstens solche Träume unterscheiden, die sinnvoll und gleichzeitig verständlich sind, d. h. eine Einreihung in unser seelisches Leben ohne
               weiteren Anstoß zulassen. Solcher Träume gibt es viele; sie sind meist kurz und
               erscheinen uns im allgemeinen wenig bemerkenswert, weil alles
               Erstaunen oder Befremden Erregende ihnen abgeht. Ihr Vorkommen ist übrigens ein
               starkes Argument gegen die Lehre, welche den Traum durch isolierte
               Tätigkeit einzelner Hirnzellgruppen entstehen läßt; es fehlen ihnen alle
               Kennzeichen herabgesetzter oder zerstückelter psychischer Tätigkeit, und doch
               erheben wir gegen ihren Charakter als Träume niemals einen Einspruch und
               verwechseln sie nicht mit den Produkten des Wachens. Eine zweite Gruppe bilden
               jene Träume, die zwar in sich zusammenhängend sind und einen klaren Sinn
               haben, aber befremdend wirken, weil wir diesen Sinn
               in
        \pend
    
         
            
            
            
        \pstart
        unserem Seelenleben nicht unterzubringen wissen. Solch ein Fall ist
               es, wenn wir z. B. träumen, daß ein lieber Verwandter an der Pest gestorben ist,
               während wir keinen Grund zu solcher Erwartung, Besorgnis oder Annahme kennen und
               uns verwundert fragen: wie komme ich zu dieser Idee? In die dritte Gruppe
               gehören endlich jene Träume, denen beides abgeht, Sinn und Verständlichkeit, die
                  unzusammenhängend, verworren und sinnlos erscheinen. Die
               überwiegende Mehrzahl der Produkte unseres Träumens zeigt diese
               Charaktere, welche die Geringschätzung der Träume und die ärztliche
               Theorie von der eingeschränkten \edtext{Seelentätigkeit}{\lemma{\textbf{Seelentätigkeit}}\Aendnote{Seelenthätigkeit E }} begründet haben. Zumal in den längeren und komplizierteren
               Traumkompositionen vermißt man nur selten die deutlichsten Zeichen der
               Inkohärenz.
        \pend
    
            
        \pstart
        Der Gegensatz von manifestem und latentem Trauminhalt hat offenbar nur für die
               Träume der zweiten, und noch eigentlicher für die der dritten Kategorie
               Bedeutung. Hier finden sich die Rätsel vor, die erst verschwinden, wenn man den
               manifesten Traum durch den latenten Gedankeninhalt ersetzt, und an einem
               Beispiel dieser Art, an einem verworrenen und unverständlichen Traum, haben wir
               auch die voranstehende Analyse ausgeführt. Wir sind aber wider unser Erwarten
               auf Motive gestoßen, die uns eine vollständige Kenntnisnahme der latenten
               Traumgedanken verwehrten, und durch die Wiederholung der gleichen Erfahrung
               dürften wir zur Vermutung geführt werden, daß zwischen dem
                  unverständlichen und verworrenen Charakter des Traumes und den Schwierigkeiten bei der
                  Mitteilung der Traumgedanken ein intimer und gesetzmäßiger Zusammenhang
                  besteht. Ehe wir die Natur dieses Zusammenhanges erforschen, werden wir
               mit Vorteil unser Interesse den leichter verständlichen Träumen der ersten
               Kategorie zuwenden, in denen manifester und latenter Inhalt zusammenfallen, die
               Traumarbeit also erspart scheint.
        \pend
    
         
            
            
            
        \pstart
        Die Untersuchung dieser Träume empfiehlt sich noch von einem anderen
               Gesichtspunkte aus. Die Träume der Kinder sind nämlich von solcher Art, also sinnvoll und nicht befremdend, was,
               nebenbei bemerkt, einen neuen Einspruch gegen die Zurückführung des
               Traumes auf dissoziierte Hirntätigkeit im Schlafe abgibt, denn warum sollte wohl
               solche Herabsetzung der psychischen Funktionen beim Erwachsenen zu
               den Charakteren des Schlafzustandes gehören, beim Kinde aber nicht? Wir dürfen
               uns aber mit vollem Recht der Erwartung hingeben, daß die Aufklärung psychischer Vorgänge beim Kinde, wo sie wesentlich vereinfacht sein mögen, sich als eine unerläßliche Vorarbeit für die Psychologie
               des Erwachsenen erweisen wird.
        \pend
    
            
        \pstart
        Ich werde also einige Beispiele von Träumen mitteilen, die ich von Kindern
               gesammelt habe: Ein Mädchen von 19 Monaten wird über einen Tag nüchtern
               erhalten, weil sie am Morgen erbrochen und sich nach der Aussage der Kinderfrau
               an Erdbeeren verdorben hat. In der Nacht nach diesem Hungertag
               hört man sie aus dem Schlafe ihren Namen nennen und dazusetzen: „Er(d)beer, Hochbeer\edtext{}{\Bendnote{
                  \textbf{Hochbeer} regionale österr. Bezeichnung für Himbeere; vgl. Österr. Wörterbuch}}
    , Eier(s)peis, Papp.“ \edtext{}{\Bendnote{
                  \textbf{Papp} regionale österr. Bezeichnung für Brei }}
     Sie träumt also,
               daß sie ißt, und hebt aus ihrem Menü gerade das hervor, was ihr die nächste
               Zeit, wie sie vermutet, karg zugemessen bleiben wird. — Ähnlich träumt von einem
               versagten Genuß ein 22monatiger Knabe, der tags zuvor seinem Onkel
               ein Körbchen mit frischen Kirschen hatte als Geschenk anbieten müssen, von
               denen er natürlich nur eine Probe kosten durfte. Er erwacht mit der
               freudigen Mitteilung: He(r)mann alle Kirschen aufgessen.
               — Ein 3¼jähriges Mädchen hatte am Tage eine Fahrt über den See gemacht, die
               ihr nicht lang genug gedauert hatte, denn sie weinte, als sie aussteigen sollte.
               Am Morgen darauf erzählte sie, daß sie in der Nacht auf dem See gefahren, die
               unterbrochene Fahrt also fortgesetzt habe. — Ein 5¼jähriger Knabe schien
               von einer Fußpartie in der Dachsteingegend wenig befriedigt; er erkundigte
               sich, so oft ein neuer Berg in Sicht kam, ob das der
        \pend
    
         
            
            
            
        \pstart
        
                  \edtext{}{\Bendnote{\textbf{Dachstein}Berg in den steirischen Ostalpen.}}
     sei,
                  und weigerte sich dann, den Weg zum Wasserfall mitzumachen. Sein
               Benehmen wurde auf Müdigkeit geschoben, erklärte sich aber besser, als er am
               nächsten Morgen seinen Traum erzählte, er sei auf den
                  Dachstein gestiegen. Er hatte offenbar erwartet, die Dachsteinbesteigung
               werde das Ziel des Ausfluges sein, und war verstimmt worden, als er den
               ersehnten Berg nicht zu Gesicht bekam. Im Traum holte er nach, was der Tag
               ihm nicht gebracht hatte. — Ganz ähnlich benahm sich der Traum eines
               sechsjährigen Mädchens, dessen Vater einen Spaziergang vor dem
               erreichten Ziele wegen vorgerückter Stunde abgebrochen hatte. Auf dem
               Rückweg war ihr eine Wegtafel aufgefallen, die einen anderen
               Ausflugsort nannte, und der Vater hatte versprochen, sie ein andermal auch
               dorthin zu führen. Sie empfing den Vater am nächsten Morgen mit der Mitteilung,
               sie habe geträumt, der Vater sei mit ihr an dem einen
                  wie an dem anderen Ort gewesen.
        \pend
    
            
        \pstart
        Das Gemeinsame dieser Kinderträume ist augenfällig. Sie erfüllen sämtlich
               Wünsche, die am Tage rege gemacht und unerfüllt geblieben sind. Sie
               sind einfache und unverhüllte Wunscherfüllungen.
        \pend
    
            
        \pstart
        Nichts anderes als eine Wunscherfüllung ist auch folgender, auf den ersten
               Eindruck nicht ganz verständlicher Kindertraum. Ein nicht vierjähriges Mädchen
               war einer poliomyelitischen Affektion wegen vom Lande in die Stadt
               gebracht worden und übernachtete bei einer kinderlosen Tante in einem
               großen — für sie natürlich übergroßen — Bette. Am nächsten Morgen
               berichtete sie, daß sie geträumt, das Bett sei ihr
                  viel zu klein gewesen, so daß sie in ihm keinen Platz gefunden. Die Lösung
               dieses Traumes als Wunschtraum ergibt sich leicht, wenn man sich erinnert, daß
                  „Großsein“ ein häufig auch geäußerter Wunsch
               der Kinder ist. Die Größe des Bettes mahnte das kleine Gernegroß allzu
               nachdrücklich an seine Kleinheit; darum korrigierte es im Traume das ihm
               unliebsame
        \pend
    
         
            
            
            
        \pstart
        Verhältnis und wurde nun so groß, daß ihm das große Bett noch zu
               klein war.
        \pend
    
            
        \pstart
        Auch wenn der Inhalt der Kinderträume sich kompliziert und verfeinert, liegt die
               Auffassung als Wunscherfüllung jedesmal sehr nahe. Ein achtjähriger Knabe
               träumt, daß er mit Achilleus im Streitwagen gefahren, den Diomedes lenkte. Er
               hat sich nachweisbar tags vorher in die Lektüre griechischer
               Heldensagen versenkt; es ist leicht zu konstatieren, daß er sich
               diese Helden zu Vorbildern genommen und bedauert hat, nicht in ihrer Zeit
               zu leben.
        \pend
    
            
        \pstart
        Aus dieser kleinen Sammlung erhellt ohne weiteres ein zweiter Charakter der
               Kinderträume, ihr Zusammenhang mit dem Tagesleben.
               Die Wünsche, die sich in ihnen erfüllen, sind vom Tage, in der Regel vom
               Vortage, erübrigt und sind im Wachdenken mit intensiver Gefühlsbetonug
               ausgestattet gewesen. Unwesentliches und Gleichgültiges, oder was dem Kinde
               so erscheinen muß, hat im Trauminhalt keine Aufnahme gefunden.
        \pend
    
            
        \pstart
        Auch bei Erwachsenen kann man zahlreiche Beispiele solcher Träume von infantilem
               Typus sammeln, die aber, wie erwähnt, meist knapp an Inhalt sind. So beantwortet
               eine Reihe von Personen einen nächtlichen Durstreiz regelmäßig mit
               dem Traume zu trinken, der also den Reiz fortzuschaffen und den Schlaf fortzusetzen strebt. Bei manchen Menschen findet man solche
               Bequemlichkeitsträume häufig vor dem Erwachen,
               wenn die Aufforderung aufzustehen, an sie herantritt. Sie träumen dann, daß
               sie schon aufgestanden sind, beim Waschtisch stehen oder sich bereits in der
               Schule, im Bureau u. dgl. befinden, wo sie zur bestimmten Zeit sein sollten. In
               der Nacht vor einer beabsichtigten Reise träumt man nicht selten, daß man
               am Bestimmungsorte angekommen ist; vor einer Theatervorstellung, einer
               Gesellschaft antizipiert der Traum nicht selten — gleichsam
               ungeduldig — das erwartete Vergnügen. Andere Male drückt
        \pend
    
         
            
            
            
        \pstart
        der Traum die Wunscherfüllung um eine Stufe indirekter aus; es
               bedarf noch der Herstellung einer Beziehung, einer Folgerung, also eines
               Beginnes von Deutungsarbeit, um die Wunscherfüllung zu erkennen. So z. B. wenn
               mir ein Mann den Traum seiner jungen Frau erzählt, daß sich bei ihr die Periode
               eingestellt habe. Ich muß daran denken, daß die junge Frau einer Gravidität
                  entgegensieht, wenn ihr die Periode ausbleibt. Dann ist die Mitteilung des Traumes eine Graviditätsanzeige, und sein Sinn ist,
               daß er den Wunsch erfüllt zeigt, die Gravidität möge doch noch eine Weile
               ausbleiben. Unter ungewöhnlichen und extremen Verhältnissen werden
               solche Träume von infantilem Charakter besonders häufig. Der Leiter einer
               Polarexpedition berichtet z. B., daß seine Mannschaft während der Überwinterung
               im Eise bei monotoner Kost und schmalen Rationen regelmäßig wie die Kinder
               von großen Mahlzeiten träumte, von Bergen von Tabak und vom Zuhausesein.
        \pend
    
            
        \pstart
        Gar nicht selten hebt sich aus einem längeren, komplizierten und im ganzen
               verworrenen Traum ein besonders klares Stück hervor, das eine unverkennbare
               Wunscherfüllung enthält, aber mit anderem, unverständlichen Material verlötet
               ist. Versucht man häufiger, auch die anscheinend undurchsichtigen Träume
               Erwachsener zu analysieren, so erfährt man zu seiner Verwunderung,
               daß diese selten so einfach sind wie die Kinderträume, und daß sie etwa hinter
               der einen Wunscherfüllung noch anderen Sinn verbergen.
        \pend
    
            
        \pstart
        Es wäre nun gewiß eine einfache und befriedigende Lösung der Traumrätsel, wenn
               etwa die Analysenarbeit uns ermöglichen sollte, auch die sinnlosen und
               verworrenen Träume Erwachsener auf den infantilen Typus der Erfüllung eines
               intensiv empfundenen Wunsches vom Tage zurückzuführen. Der Anschein
               spricht gewiß nicht für diese Erwartung. Die Träume sind meist voll des
               gleichgültigsten und fremdartigsten Materials, und von Wunscherfüllung ist in
               ihrem Inhalt nichts zu merken.
        \pend
    
         
            
            
            
        \pstart
        Ehe wir aber die infantilen Träume, die unverhüllte Wunscherfüllungen
               sind, verlassen, wollen wir nicht versäumen, einen längst bemerkten
               Hauptcharakter des Traumes zu erwähnen, der gerade in dieser Gruppe am reinsten
               hervortritt. Ich kann jeden dieser Träume durch einen Wunschsatz ersetzen: Oh,
               hätte die Fahrt auf dem See doch länger gedauert; — wäre ich doch schon
               gewaschen und angezogen; — hätte ich doch die Kirschen behalten dürfen, anstatt
               sie dem Onkel zu geben; aber der Traum gibt mehr als diesen Optativ. Er zeigt
               den Wunsch als bereits erfüllt, stellt diese Erfüllung als real und gegenwärtig
               dar, und das Material der Traumdarstellung besteht vorwiegend — wenn auch
               nicht ausschließlich — aus Situationen und meist visuellen Sinnesbildern. Auch
               in dieser Gruppe wird also eine Art Umwandlung — die man als Traumarbeit
               bezeichnen darf — nicht völlig vermißt: Ein im
                  Optativ stehender Gedanke ist durch eine Anschauung im Präsens ersetzt.
        \pend
    
         
            
            
            \pstart\eledsection*{IV}\pend
            
        \pstart
        Wir werden geneigt sein anzunehmen, daß eine solche Umsetzung in eine
               Situation auch bei den verworrenen Träumen stattgefunden hat, wiewohl wir nicht
               wissen können, ob sie auch hier einen Optativ betraf. Das eingangs mitgeteilte
                  Traumbeispiel, in dessen Analyse wir ein Stück weit eingegangen
               sind, gibt uns allerdings an zwei Stellen Anlaß, etwas Derartiges zu
               vermuten. Es kommt in der Analyse vor, daß meine Frau sich an der Tafel mit
               anderen beschäftigt, was ich als unangenehm empfinde; der Traum enthält davon
               das genaue Gegenteil, daß die Person, die meine
               Frau ersetzt, sich ganz mir zuwendet. Zu welchem Wunsch kann aber ein
               unangenehmes Erlebnis besser Anlaß geben, als zu dem, daß sich das Gegenteil
               davon ereignet haben sollte, wie es der Traum als vollzogen enthält? In
               ganz ähnlichem Verhältnis steht der bittere Gedanke in der Analyse, daß ich
               nichts umsonst gehabt habe, zu der Rede der Frau im Traum: Sie haben ja immer so
               schöne Augen gehabt. Ein Teil der Gegensätzlichkeiten zwischen manifestem
               und latentem Trauminhalt dürfte sich also auf Wunscherfüllung zurückführen
               lassen.
        \pend
    
            
        \pstart
        Augenfälliger ist aber eine andere Leistung der Traumarbeit, durch welche die
               inkohärenten Träume zustande kommen. Vergleicht man an einem
               beliebigen Beispiel die Zahl der Vorstellungselemente oder den Umfang
               der Niederschrift beim Traum und bei den Traumgedanken, zu denen die Analyse
        \pend
    
         
            
            
            
        \pstart
        führt, und von denen man eine Spur im Traume wiederfindet, so kann
               man nicht bezweifeln, daß die Traumarbeit hier eine großartige Zusammendrängung
               oder Verdichtung zustande gebracht hat. Über das
               Ausmaß dieser Verdichtung kann man sich zunächst ein Urteil nicht bilden; sie
               imponiert aber um so mehr, je tiefer man in die Traumanalyse eingedrungen ist.
               Da findet man dann kein Element des Trauminhaltes, von dem die
               Assoziationsfäden nicht nach zwei oder mehr Richtungen auseinandergingen, keine Situation, die nicht aus zwei oder mehr Eindrücken und
               Erlebnissen zusammengestückelt wäre. Ich träumte z. B. einmal von einer Art
               Schwimmbassin, in dem die Badenden nach allen Richtungen auseinanderfuhren; an
               einer Stelle des Randes stand eine Person, die sich zu einer badenden
               Person neigte, wie um sie herauszuziehen. Die Situation war zusammengesetzt aus
               der Erinnerung an ein Erlebnis der Pubertätszeit und aus zwei
               Bildern, von denen ich eines kurz vor dem Traum gesehen hatte. Die zwei Bilder
               waren das der Überraschung im Bade aus dem Schwindschen Zyklus Melusine (siehe die
               auseinanderfahrenden Badenden) und ein Sintflutbild eines italienischen
               Meisters. Das kleine Erlebnis aber hatte darin bestanden, daß ich zusehen
               konnte, wie in der Schwimmschule der Bademeister einer Dame aus dem Wasser half,
               die sich bis zum Eintritt der Herrenstunde verspätet hatte. — Die Situation
               in dem zur Analyse gewählten Beispiel leitet mich bei der Analyse auf
               eine kleine Reihe von Erinnerungen, von denen jede zum Trauminhalt etwas
               beigesteuert hat. Zunächst ist es die kleine Szene aus der Zeit meiner Werbung,
               von der ich bereits gesprochen; ein Händedruck unter dem Tisch, der damals
               vorfiel, hat für den Traum das Detail „unter dem Tisch“, das ich der
               Erinnerung nachträglich einfügen muß, geliefert. Von „Zuwendung“ war
               natürlich damals keine Rede; ich weiß aus der Analyse, daß dieses
               Element die Wunscherfüllung durch Gegensatz ist, die zum Benehmen meiner Frau an
               der Table d’hôte gehört.
        \pend
    
         
            
            
            
        \pstart
        Hinter dieser rezenten Erinnerung verbirgt sich aber eine ganz
               ähnliche und viel bedeutsamere Szene aus unserer Verlobungszeit, die uns für
               einen ganzen Tag entzweite. Die Vertraulichkeit, die Hand auf das Knie zu legen,
               gehört in einen ganz verschiedenen Zusammenhang und zu ganz anderen Personen.
               Dieses Traumelement wird selbst wieder zum Ausgangspunkt zweier
               besonderer Erinnerungsreihen usw.
        \pend
    
            
        \pstart
        Das Material aus den Traumgedanken, welches zur Bildung der Traumsituation
               zusammengeschoben wird, muß natürlich für diese Verwendung von vorneherein
               brauchbar sein. Es bedarf hiezu eines — oder mehrerer — in allen Komponenten
                  vorhandenen Gemeinsamen. Die
               Traumarbeit verfährt dann wie
               Francis Galton bei der Herstellung seiner
                  Familienphotographien. Sie bringt die verschiedenen Komponenten
               wie übereinander gelegt zur Deckung; dann tritt das Gemeinsame
               im Gesamtbild deutlich hervor, die widersprechenden Details löschen
               einander nahezu aus. Dieser Herstellungsprozeß erklärt auch zum Teil die
               schwankenden Bestimmungen von eigentümlicher Verschwommenheit so
               vieler Elemente des Trauminhalts. Die Traumdeutung spricht, auf
               dieser Einsicht fußend, folgende Regel aus: Wo sich bei der Analyse eine Unbestimmtheit noch in ein entweder — oder
               auflösen läßt, da ersetze man dies für die Deutung durch ein „und“ und nehme jedes Glied der scheinbaren
               Alternative zum unabhängigen Ausgang einer Reihe von Einfällen.
        \pend
    
            
        \pstart
        Wo solche Gemeinsame zwischen den Traumgedanken
               nicht vorhanden sind, da bemüht sich die Traumarbeit solche zu schaffen, um die gemeinsame Darstellung im Traume zu
               ermöglichen. Der bequemste Weg, um zwei Traumgedanken, die noch nichts
               Gemeinsames haben, einander näher zu bringen, besteht in der Veränderung des
               sprachlichen Ausdrucks für den einen, wobei ihm etwa noch der andere durch eine
               entsprechende Umgießung in einen anderen Ausdruck entgegenkommt. Es ist
        \pend
    
         
            
            
            
        \pstart
        das ein ähnlicher Vorgang wie beim Reimeschmieden, wobei der
               Gleichklang das gesuchte Gemeinsame ersetzt. Ein gutes Stück der Traumarbeit
               besteht in der Schöpfung solcher häufig sehr witzig, oft aber gezwungen
               erscheinenden Zwischengedanken, welche von der gemeinsamen Darstellung im
               Trauminhalt bis zu den nach Form und Wesen verschiedenen, durch die Traumanlässe motivierten Traumgedanken reichen. Auch in der Analyse
               unseres Traumbeispiels finde ich einen derartigen Fall von Umformung
               eines Gedankens zum Zwecke des Zusammentreffens mit einem anderen, ihm
               wesensfremden. Bei der Fortsetzung der Analyse stoße ich nämlich auf den
               Gedanken: Ich möchte auch einmal etwas umsonst haben;
               aber diese Form ist für den Trauminhalt nicht brauchbar. Sie wird darum
               durch eine neue ersetzt: Ich möchte gerne etwas genießen ohne „Kosten“ zu haben. Das
               Wort Kosten paßt nun mit seiner zweiten
               Bedeutung in den Vorstellungskreis der Table d’hôte und kann seine Darstellung
               durch den im Traum aufgetischten Spinat finden.
               Wenn bei uns eine Speise zu Tische kommt, welche von den Kindern abgelehnt wird,
               so versucht es die Mutter wohl zuerst mit Milde und fordert von den
               Kindern: Nur ein bißchen kosten. Daß die Traumarbeit die Zweideutigkeit der Worte so unbedenklich ausnützt,
               erscheint zwar sonderbar, stellt sich aber bei reicherer Erfahrung als ein ganz
               gewöhnliches Vorkommnis heraus.
        \pend
    
            
        \pstart
        Durch die Verdichtungsarbeit des Traumes erklären sich auch gewisse Bestandteile
               seines Inhaltes, die nur ihm eigentümlich sind und im wachen Vorstellen nicht
               gefunden werden. Es sind dies die Sammel- und
                  Mischpersonen und die sonderbaren
                  Mischgebilde, Schöpfungen, den
               Tierkompositionen orientalischer Völkerphantasie vergleichbar, die aber in
               unserem Denken bereits zu Einheiten erstarrt sind, während die Traumkompositionen in unerschöpflichem Reichtum immer neu gebildet
               werden. Jeder kennt solche Gebilde aus seinen eigenen Träumen;
        \pend
    
         
            
            
            
        \pstart
        die Weisen ihrer Herstellung sind sehr mannigfaltig. Ich kann eine
               Person zusammensetzen, indem ich ihr Züge von der einen und von der anderen
               verleihe, oder indem ich ihr die Gestalt der einen gebe und dabei im Traum den
               Namen der anderen denke, oder ich kann die eine Person visuell vorstellen, sie
               aber in eine Situation versetzen, die sich mit der anderen ereignet hat. In
               all diesen Fällen ist die Zusammenziehung verschiedener Personen zu einem
               einzigen Vertreter im Trauminhalt sinnvoll, sie soll ein „und“, „gleichwie“,
               eine Gleichstellung der originalen Personen in einer gewissen Hinsicht bedeuten,
               die auch im Traum selbst erwähnt sein kann. In der Regel aber ist diese
               Gemeinsamkeit der verschmolzenen Personen erst durch die Analyse aufzusuchen und
               wird im Trauminhalt eben bloß durch die Bildung der Sammelperson angedeutet.
        \pend
    
            
        \pstart
        Dieselbe Mannigfaltigkeit der Herstellungsweise und die nämliche Regel bei der
               Auflösung gilt auch für die unermeßlich reichhaltigen Mischgebilde
               des Trauminhaltes, von denen ich Beispiele wohl nicht anzuführen brauche. Ihre
               Sonderbarkeit verschwindet ganz, wenn wir uns entschließen, sie nicht in eine
               Reihe mit den Objekten der Wahrnehmung im Wachen zu stellen, sondern uns
               erinnern, daß sie eine Leistung der Traumverdichtung darstellen und
               in treffender Abkürzung einen gemeinsamen Charakter der so kombinierten Objekte
               hervorheben. Die Gemeinsamkeit ist auch hier meist aus der Analyse einzusetzen.
               Der Trauminhalt sagt gleichsam nur aus: Alle diese
                  Dinge haben ein X gemeinsam. Die Zersetzung solcher Mischgebilde durch
               die Analyse führt oft auf dem kürzesten Weg zur Bedeutung des Traumes. So
               träumte ich einmal, daß ich mit einem meiner früheren Universitätslehrer in
               einer Bank sitze, die mitten unter anderen Bänken eine rasch fortschreitende
               Bewegung erfährt. Es war dies eine Kombination von Hörsaal und Trottoir
               roulant. Die weitere Verfolgung des Gedankens übergehe ich. — Ein andermal
               sitze ich im Waggon und halte auf dem Schoß einen
        \pend
    
         
            
            
            
        \pstart
        Gegenstand von der Form eines Zylinderhutes, der aber aus durchsichtigem Glas besteht Die Situation läßt mir sofort das Sprichwort einfallen: Mit dem Hute in der Hand kommt man durchs ganze
               Land. Der Glaszylinder erinnert auf kurzen Umwegen an das Auersche Licht, und ich
               weiß bald, daß ich eine Erfindung machen möchte, die mich so reich und
               unabhängig werden läßt wie meinen Landsmann, den Dr. Auer von Welsbach, die seinige, und daß ich dann Reisen machen
               will, anstatt in Wien zu bleiben. Im Traume reise ich mit meiner Erfindung
               — dem allerdings noch nicht gebräuchlichen Hutzylinder aus Glas. — Ganz
               besonders liebt es die Traumarbeit zwei in gegensätzlicher Beziehung
               stehende Vorstellungen durch das nämliche Mischgebilde darzustellen, so z. B.
               wenn eine Frau sich im Traume, einen haben Blumenstengel tragend, sieht, wie der
               Engel auf den Bildern von Mariä Verkündigung dargestellt wird (Unschuld —
               Marie ist ihr eigener Name), der Stengel aber mit dicken, weißen Blüten besetzt
               ist, die Kamelien gleichen (Gegensatz zu Unschuld: Kameliendame).
        \pend
    
            
        \pstart
        Ein gutes Stück dessen, was wir über die Traumverdichtung erfahren haben, läßt
               sich in der Formel zusammenfassen: Jedes der Elemente des Trauminhaltes ist
               durch das Material der Traumgedanken überdeterminiert, führt seine Abstammung nicht auf ein einzelnes
               Element der Traumgedanken, sondern auf eine ganze Reihe von solchen zurück, die
               einander in den Traumgedanken keineswegs nahe stehen müssen, sondern den
               verschiedensten Bezirken des Gedankengewebes angehören können. Das Traumelement
               ist im richtigen Sinne die Vertretung im
                  Trauminhalt für all dies disparate Material. Die Analyse deckt aber noch
               eine andere Seite der zusammengesetzten Beziehung zwischen Trauminhalt und
               Traumgedanken auf. So wie von jedem Traumelement Verbindungen zu
               mehreren Traumgedanken führen, so ist auch in der Regel ein Traumgedanke durch mehr als ein
        \pend
    
         
            
            
            
        \pstart
        Traumelement vertreten; die
               Assoziationsfäden konvergieren nicht einfach von den Traumgedanken
               bis zum Trauminhalt, sondern überkreuzen und durchweben sich
               vielfach unterwegs.
        \pend
    
            
        \pstart
        Neben der Verwandlung eines Gedankens in eine Situation (der „Dramatisierung“)
               ist die Verdichtung der wichtigste und eigentümlichste Charakter der
               Traumarbeit. Von einem Motiv, welches zu solcher Zusammendrängung des Inhaltes
               nötigen würde, ist uns aber zunächst nichts enthüllt worden.
        \pend
    
         
            
            
            \pstart\eledsection*{V}\pend
            
        \pstart
        Bei den komplizierten und verworrenen Träumen, die uns jetzt beschäftigen, läßt
               sich nicht der ganze Eindruck von Unähnlichkeit zwischen Trauminhalt und
               Traumgedanken auf Verdichtung und Dramatisierung zurückführen. Es liegen Zeugnisse für die Wirksamkeit eines dritten Faktors vor, die einer
               sorgfältigen Sammlung würdig sind.
        \pend
    
            
        \pstart
        Ich merke vor allem, wenn ich durch Analyse zur Kenntnis der Traumgedanken
               gelangt bin, daß der manifeste Trauminhalt ganz andere Stoffe behandelt als der
               latente. Dies ist freilich nur ein Schein, der sich bei genauerer Untersuchung
               verflüchtigt, denn schließlich finde ich allen Trauminhalt in den Traumgedanken ausgeführt, fast alle Traumgedanken durch den Trauminhalt vertreten wieder. Aber es bleibt von der Verschiedenheit
               doch etwas bestehen. Was in dem Traum breit und deutlich als der wesentliche
               Inhalt hingestellt war, das muß sich nach der Analyse mit einer höchst
               untergeordneten Rolle unter den Traumgedanken begnügen, und was nach
               der Aussage meiner Gefühle unter den Traumgedanken auf die größte Beachtung
               Anspruch hat, dessen Vorstellungsmaterial findet sich im Trauminhalt entweder gar nicht vor oder ist durch eine entfernte Anspielung in
               einer undeutlichen Region des Traumes vertreten. Ich kann diese Tatsache so
                  beschreiben: Während der Traumarbeit übergeht die
                  psychische Intensität von den Gedanken und Vorstellungen, denen sie
                  berechtigter-
        \pend
    
         
            
            
            
        \pstart
        weise zukommt, auf andere, die nach meinem
                  Urteil keinen Anspruch auf solche Betonung haben. Kein anderer Vorgang trägt
               soviel dazu bei, um den Sinn des Traumes zu verbergen und mir den
               Zusammenhang von Trauminhalt und Traumgedanken unkenntlich zu machen.
               Während dieses Vorganges, den ich die Traumverschiebung nennen
               will, sehe ich auch die psychische Intensität, Bedeutsamkeit oder
               Affektfähigkeit von Gedanken sich in sinnliche Lebhaftigkeit
               umsetzen. Das Deutlichste im Trauminhalt erscheint mir ohne weiteres als das
               Wichtigste; gerade in einem undeutlichen Traumelement kann ich aber oft den
               direktesten Abkömmling des wesentlichen Traumgedankens erkennen.
        \pend
    
            
        \pstart
        Was ich Traumverschiebung genannt habe, könnte ich auch als Umwertung der psychischen Wertigkeiten
               bezeichnen. Ich habe aber das Phänomen nicht erschöpfend gewürdigt,
               wenn ich nicht hinzufüge, daß diese Verschiebungs- oder Umwertungsarbeit an den
               einzelnen Träumen mit einem sehr wechselnden Betrag beteiligt ist. Es gibt
               Träume, die fast ohne jede Verschiebung zustande gekommen sind. Diese sind
               gleichzeitig die sinnvollen und verständlichen, wie wir z. B. die
               unverhüllten Wunschträume kennen gelernt haben. In anderen Träumen hat
               nicht mehr ein Stück der Traumgedanken den ihm eigenen psychischen Wert
               behalten, oder zeigt sich alles Wesentliche aus den Traumgedanken durch
               Nebensächliches ersetzt, und dazwischen läßt sich die vollständigste Reihe von
               Übergängen erkennen. Je dunkler und verworrener ein Traum ist, desto
               größeren Anteil darf man dem Moment der Verschiebung an seiner Bildung
               zuschreiben.
        \pend
    
            
        \pstart
        Unser zur Analyse gewähltes Beispiel zeigt wenigstens soviel von Verschiebung,
               daß sein Inhalt anders zentriert erscheint als
               die Traumgedanken. In den Vordergrund des Trauminhaltes drängt sich eine
               Situation, als ob eine Frau mir Avancen machen würde; das Hauptgewicht in den
               Traumgedanken ruht auf dem
        \pend
    
         
            
            
            
        \pstart
        Wunsche, einmal uneigennützige Liebe, die „nichts kostet“, zu
               genießen, und diese Idee ist hinter der Redensart von den schönen Augen und der
               entlegenen Anspielung „Spinat“ versteckt.
        \pend
    
            
        \pstart
        Wenn wir durch die Analyse die Traumverschiebung rückgängig machen,
               gelangen wir zu vollkommen sicher lautenden Auskünften über zwei vielumstrittene
               Traumprobleme, über die Traumerreger und über den Zusammenhang des Traumes
               mit dem Wachleben. Es gibt Träume, die ihre Anknüpfung an die Erlebnisse
               des Tages unmittelbar verraten; in anderen ist von solcher Beziehung keine Spur
               zu entdecken. Nimmt man dann die Analyse zu Hilfe, so kann man zeigen, daß jeder
               Traum ohne mögliche Ausnahme an einen Eindruck der letzten Tage —
               wahrscheinlich ist es richtiger, zu sagen: des letzten Tages vor dem Traum (des
               Traumtages) — anknüpft. Der Eindruck, welchem die Rolle des Traumerregers
               zufällt, kann ein so bedeutsamer sein, daß uns die Beschäftigung mit ihm im
               Wachen nicht Wunder nimmt, und in diesem Falle sagen wir vom Traume mit
               Recht aus, er setze die wichtigen Interessen des Wachlebens fort. Gewöhnlich
               aber, wenn sich in dem Trauminhalt eine Beziehung zu einem Tageseindruck
               vorfindet, ist dieser so geringfügig, bedeutungslos und des
               Vergessens würdig, daß wir uns an ihn selbst nicht ohne einige Mühe besinnen
               können. Der Trauminhalt selbst scheint sich dann, auch wo er
               zusammenhängend und verständlich ist, mit den gleichgültigsten Lappalien zu
               beschäftigen, die unseres Interesses im Wachen unwürdig wären. Ein gutes Stück
               der Mißachtung des Traumes leitet sich von dieser Bevorzugung des Gleichgültigen
               und Nichtigen im Trauminhalte her.
        \pend
    
            
        \pstart
        Die Analyse zerstört den Schein, auf den sich dieses geringschätzige
               Urteil gründet. Wo der Trauminhalt einen indifferenten Eindruck als Traumerreger
               in den Vordergrund stellt, da weist die Analyse regelmäßig das bedeutsame, mit
               Recht aufregende Erlebnis nach, welches sich durch das gleichgültige ersetzt,
               mit
        \pend
    
         
            
            
            
        \pstart
        dem es ausgiebige assoziative Verbindungen eingegangen hat. Wo der
               Trauminhalt bedeutungsloses und uninteressantes Vorstellungsmaterial
               behandelt, da deckt die Analyse die zahlreichen Verbindungswege auf,
               mittelst welcher dies Wertlose mit dem Wertvollsten in der
               psychischen Schätzung des Einzelnen zusammenhängt. Es sind nur Akte der Verschiebungsarbeit wenn anstatt des
                  mit Recht erregenden Eindruckes der indifferente, anstatt des mit Recht
                  interessanten Materials das
                  gleichgültige zur Aufnahme in den
                  Trauminhalt gelangen. Beantwortet man die Fragen nach den Traumerregern
               und nach dem Zusammenhang des Träumens mit dem täglichen Treiben
               nach den neuen Einsichten, die man bei der Ersetzung des manifesten
               Trauminhaltes durch den latenten gewonnen hat, so muß man sagen: der Traum beschäftigt sich niemals mit Dingen, die uns
                  nicht auch bei Tag zu beschäftigen würdig sind, und Kleinigkeiten, die uns bei Tag
                  nicht anfechten, vermögen es auch nicht, uns in den Schlaf zu verfolgen.
        \pend
    
            
        \pstart
        Welches ist der Traumerreger in dem zur Analyse gewählten Beispiel? Das wirklich
               bedeutungslose Erlebnis, daß mir ein Freund zu einer kostenlosen Fahrt im Wagen verhalf. Die Situation der Table d’hôte im
               Traum enthält eine Anspielung auf diesen indifferenten Anlaß, denn ich hatte im
               Gespräch den Taxameterwagen in Parallele zur Table d’hôte gebracht. Ich
               kann aber auch das bedeutsame Erlebnis angeben, welches sich durch dieses
               kleinliche vertreten läßt. Wenige Tage vorher hatte ich eine größere Geldausgabe
               für eine mir teuere Person meiner Familie gemacht. Kein Wunder, heißt es in den
               Traumgedanken, wenn diese Person mir dafür dankbar wäre, diese Liebe wäre
               nicht „kostenlos“. Kostenlose Liebe steht aber unter den Traumgedanken im Vordergrunde. Daß ich vor nicht langer Zeit mehrere Wagenfahrten mit dem betreffenden Verwandten
        \pend
    
         
            
            
            
        \pstart
        gemacht, setzt die eine Wagenfahrt mit meinem Freund in den Stand,
               mich an die Beziehungen zu jener anderen Person zu erinnern. — Der indifferente
               Eindruck, der durch derartige Verknüpfungen zum Traumerreger wird,
               unterliegt noch einer Bedingung, die für die wirkliche Traumquelle nicht gilt;
               er muß jedesmal ein rezenter sein, vom Traumtage
               herrühren.
        \pend
    
            
        \pstart
        Ich kann das Thema der Traumverschiebung nicht verlassen, ohne eines
               merkwürdigen Vorganges bei der Traumbildung zu gedenken, bei dem Verdichtung und
               Verschiebung zum Effekt zusammenwirken. Wir haben schon bei der Verdichtung
               den Fall kennen gelernt, daß sich zwei Vorstellungen in den Traumgedanken, die etwas Gemeinsames, einen Berührungspunkt haben, im Trauminhalt
               durch eine Mischvorstellung ersetzen, in der ein deutlicherer Kern dem
               Gemeinsamen, undeutliche Nebenbestimmungen den Besonderheiten der
               beiden entsprechen. Tritt zu dieser Verdichtung eine Verschiebung hinzu, so
               kommt es nicht zur Bildung einer Mischvorstellung, sondern eines mittleren Gemeinsamen, das sich ähnlich zu den
               einzelnen Elementen verhält wie die Resultierende im Kräfteparallelogramm zu
               ihren Komponenten. Im Inhalt eines meiner Träume ist z. B. von einer
               Injektion mit Propylen die Rede. In der Analyse
               gelange ich zunächst nur zu einem indifferenten, als Traumerreger wirksamen Erlebnis, bei welchem Amylen
               eine Rolle spielt. Die Vertauschung von Amylen mit Propylen kann ich noch
               nicht rechtfertigen. Zu dem Gedankenkreis desselben Traumes gehört aber
               auch die Erinnerung an einen ersten Besuch in München, wo mir die Propyläen auffielen. Die näheren Umstände der
               Analyse legen es nahe anzunehmen, daß die Einwirkung dieses zweiten
               Vorstellungskreises auf den ersten die Verschiebung von Amylen auf Propylen
               verschuldet hat. Propylen ist sozusagen die
               Mittelvorstellung zwischen Amylen und Propyläen und ist darum nach Art eines Kompromisses durch gleichzeitige Verdichtung und
               Verschiebung in den Trauminhalt gelangt.
        \pend
    
         
            
            
            
        \pstart
        Dringender noch als bei der Verdichtung äußert sich hier bei der
               Verschiebungsarbeit das Bedürfnis, ein Motiv für diese rätselhaften
               Bemühungen der Traumarbeit aufzufinden.
        \pend
    
         
            
            
            \pstart\eledsection*{VI}\pend
            
        \pstart
        Ist es hauptsächlich der Verschiebungsarbeit zur Last zu legen, wenn man die
               Traumgedanken im Trauminhalt nicht wiederfindet oder nicht
               wiedererkennt, — ohne daß man das Motiv solcher Entstellung errät, — so führt
               eine andere und gelindere Art der Umwandlung, welche mit den Traumgedanken
                  vorgenommen wird, zur Aufdeckung einer neuen, aber leichtverständlichen Leistung der Traumarbeit. Die nächsten Traumgedanken, welche man durch die Analyse entwickelt, fallen nämlich
               häufig durch ihre ungewöhnliche Einkleidung auf; sie scheinen nicht in den
               nüchternen sprachlichen Formen gegeben, deren sich unser Denken am liebsten
               bedient, sondern sind vielmehr in symbolischer Weise durch
               Gleichnisse und Metaphern, wie in bilderreicher Dichtersprache, dargestellt. Es
               ist nicht schwierig, für diesen Grad von Gebundenheit im Ausdruck der
               Traumgedanken die Motivierung zu finden. Der Trauminhalt besteht zumeist aus
               anschaulichen Situationen; die Traumgedanken müssen also vorerst eine Zurichtung
               erfahren, welche sie für diese Darstellungsweise brauchbar macht. Man stelle
               sich etwa vor die Aufgabe, die Sätze eines politischen Leitartikels oder
               eines Plaidoyers im Gerichtssaal durch eine Folge von Bilderzeichnungen zu
               ersetzen, und man wird dann leicht die Veränderungen verstehen, zu
               welcher die Rücksicht auf Darstellbarkeit im
                  Trauminhalt die Traumarbeit nötigt.
        \pend
    
            
        \pstart
        Unter dem psychischen Material der Traumgedanken befinden
        \pend
    
         
            
            
            
        \pstart
        sich \edtext{regelmäßig}{\lemma{\textbf{regelmäßig}}\Aendnote{regelmässig E }} Erinnerungen an eindrucksvolle Erlebnisse, — nicht selten aus früher
               Kindheit \edtext{, —}{\lemma{\textbf{, —}}\Aendnote{— E }} die also selbst als Situationen mit meist visuellem Inhalt \edtext{erfaßt}{\lemma{\textbf{erfaßt}}\Aendnote{erfasst ED1D2 }} worden sind. Wo es irgend möglich ist, \edtext{äußert}{\lemma{\textbf{äußert}}\Aendnote{äussert ED1D2 }} dieser \edtext{Bestandteil}{\lemma{\textbf{Bestandteil}}\Aendnote{Bestandtheil E }} der Traumgedanken einen bestimmenden \edtext{Einfluß}{\lemma{\textbf{Einfluß}}\Aendnote{Einfluss ED1 Einfluss D2 }} auf die Gestaltung des Trauminhalts, indem er gleichsam als \edtext{Kristallisationspunkt}{\lemma{\textbf{Kristallisationspunkt}}\Aendnote{Krystallisationspunkt ED1 }} anziehend und \edtext{verteilend}{\lemma{\textbf{verteilend}}\Aendnote{vertheilend E }} auf das Material der Traumgedanken wirkt. Die Traumsituation
               ist oft nichts anderes als eine \edtext{modifizierte}{\lemma{\textbf{modifizierte}}\Aendnote{modifizirte ED1 }} und durch Einschaltungen \edtext{komplizierte}{\lemma{\textbf{komplizierte}}\Aendnote{komplizirte E }} Wiederholung eines solchen eindrucksvollen Erlebnisses;
               getreue und unvermengte Reproduktionen realer
               \edtext{Szenen}{\lemma{\textbf{Szenen}}\Aendnote{Scenen E }} bringt der Traum hingegen nur \edtext{sehr}{\lemma{\textbf{sehr}}\Aendnote{höchst E }} selten.
        \pend
    
            
        \pstart
        Der Trauminhalt besteht aber nicht \edtext{ausschließlich}{\lemma{\textbf{ausschließlich}}\Aendnote{ausschliesslich ED1D2 }} aus Situationen, sondern \edtext{schließt}{\lemma{\textbf{schließt}}\Aendnote{schliesst ED1D2 }} auch unvereinigte Brocken von visuellen Bildern, Reden und selbst Stücke
               von unveränderten Gedanken ein. Es wird daher vielleicht anregend wirken,
               wenn wir in knappster Weise die Darstellungsmittel mustern, welche der
               Traumarbeit zur Verfügung stehen, um in der
               \edtext{eigentümlichen}{\lemma{\textbf{eigentümlichen}}\Aendnote{eigenthümlichen E }} Ausdrucksweise des Traumes die Traumgedanken wiederzugeben.
        \pend
    
            
        \pstart
        Die Traumgedanken, welche wir durch die Analyse erfahren, zeigen sich uns als
               ein psychischer Komplex \edtext{von}{\lemma{\textbf{von}}\Aendnote{vom ED1D2 }} allerverwickeltstem Aufbau. Die Stücke desselben stehen in den
               mannigfaltigsten logischen Relationen \edtext{zu einander}{\lemma{\textbf{zu einander}}\Aendnote{zueinander S }}; sie bilden \edtext{Vorder- und}{\lemma{\textbf{Vorder- und}}\Aendnote{Vorderund S }} Hintergrund, Bedingungen, Abschweifungen, Erläuterungen,
               Beweisgänge und Einsprüche. Fast \edtext{regelmäßig}{\lemma{\textbf{regelmäßig}}\Aendnote{regelmässig E }} steht neben einem Gedankengang sein kontradiktorisches
               Widerspiel. Diesem Material fehlt keiner der Charaktere, die uns von unserem
               wachen Denken her bekannt sind. Soll nun aus alledem ein Traum werden, so
                  unterliegt dies psychische Material einer Pressung, die es
               ausgiebig verdichtet, einer inneren Zerbröckelung und Verschiebung, welche
               gleichsam neue Oberflächen schafft, und einer auswählenden Einwirkung durch die
               zur Situationsbildung tauglichsten Bestandteile. Mit Rücksicht auf
               die Genese dieses Materials verdient ein
        \pend
    
         
            
            
            
        \pstart
        solcher Vorgang den Namen einer „Regression“. Die logischen Bande, welche das psychische Material bisher
               zusammengehalten hatten, gehen nun aber bei dieser Umwandlung zum
               Trauminhalt verloren. Die Traumarbeit übernimmt gleichsam nur den sachlichen Inhalt der Traumgedanken zur Bearbeitung. Der Analysenarbeit bleibt es überlassen, den Zusammenhang herzustellen, den
               die Traumarbeit vernichtet hat.
        \pend
    
            
        \pstart
        Die Ausdrucksmittel des Traumes sind also kümmerlich zu nennen im Vergleich zu
               denen unserer Denksprache, doch braucht der Traum auf die Wiedergabe der
               logischen Relationen unter den Traumgedanken nicht völlig zu verzichten; es
               gelingt ihm vielmehr häufig genug, dieselben durch formale Charaktere
               seines eigenen Gefüges zu ersetzen.
        \pend
    
            
        \pstart
        Der Traum wird zunächst dem unleugbaren Zusammenhang zwischen allen Stücken der
               Traumgedanken dadurch gerecht, daß er dieses Material zu einer Situation
               vereinigt. Er gibt logischen Zusammenhang wieder
               als Annäherung in Zeit und Raum, ähnlich wie der
               Maler, der alle Dichter zum Bild des Parnaß zusammenstellt, die niemals auf
               einem Berggipfel beisammen gewesen sind, wohl aber begrifflich eine
               Gemeinschaft bilden. Er setzt diese Darstellungsweise ins Einzelne fort,
               und oft, wenn er zwei Elemente nahe bei einander im Trauminhalt zeigt,
               bürgt er für einen besonders innigen Zusammenhang zwischen ihren Entsprechenden
               in den Traumgedanken. Es ist hier übrigens zu bemerken, daß alle in derselben
               Nacht produzierten Träume bei der Analyse ihre Herkunft aus dem nämlichen
               Gedankenkreis erkennen lassen.
        \pend
    
            
        \pstart
        Die Kausalbeziehung zwischen zwei Gedanken wird
                  entweder ohne Darstellung gelassen oder ersetzt durch das Nacheinander von zwei verschieden langen Traumstücken. Häufig ist diese
               Darstellung eine verkehrte, indem der Anfang des Traumes die Folgerung, der
               Schluß desselben die Voraussetzung bringt. Die direkte Verwandlung eines Dinges in ein anderes
        \pend
    
         
            
            
            
        \pstart
        im Traum scheint die Relation von Ursache
               und Wirkung darzustellen.
        \pend
    
            
        \pstart
        Die Alternative „Entweder—Oder“ drückt der Traum
               niemals aus, sondern nimmt ihre beiden Glieder wie gleichberechtigt
               in den nämlichen Zusammenhang auf. Daß ein Entweder—Oder, welches bei
               der Traumreproduktion gebraucht wird, durch „Und“ zu übersetzen ist, habe ich bereits erwähnt.
        \pend
    
            
        \pstart
        Vorstellungen, die im Gegensatz zu einander stehen, werden mit Vorliebe im
               Traume durch das nämliche Element ausgedrückt.1 Das „nicht“ scheint für den Traum nicht zu existieren. Opposition
               zwischen zwei Gedanken, die Relation der Umkehrung,
               findet eine höchst bemerkenswerte Darstellung im Traum. Sie wird dadurch
               ausgedrückt, daß ein anderes Stück des Trauminhaltes — gleichsam wie
               nachträglich — in sein Gegenteil verkehrt wird. Eine andere Art, Widerspruch auszudrücken, werden wir später
               kennen lernen. Auch die im Traum so häufige Sensation der gehemmten Bewegung dient dazu, einen Widerspruch
               zwischen Impulsen, einen Willenskonflikt,
               darzustellen.
        \pend
    
            
        \pstart
        Einer einzigen unter den logischen Relationen, der der Ähnlichkeit, Gemeinsamkeit,
                  Übereinstimmung, kommt der Mechanismus der Traumbildung im höchsten
               Ausmaße zugute. Die Traumarbeit bedient sich dieser Fälle als Stützpunkte für
               die Traumverdichtung, indem sie alles, was solche Übereinstimmung zeigt, zu
               einer neuen Einheit zusammenzieht.
        \pend
    
            
        \pstart
        Diese kurze Reihe von groben Bemerkungen reicht natürlich nicht aus, um die
               ganze Fülle der formalen Darstellungsmittel des Traumes für die logischen
               Relationen der Traumgedanken zu würdigen. Die einzelnen Träume sind in dieser
               Hinsicht feiner
        \pend
    
            
        \footnote{1) Es ist bemerkenswert, daß
               namhafte Sprachforscher behaupten, die ältesten menschlichen Sprachen hätten
               ganz allgemein kontradiktorische Gegensätze durch das nämliche Wort zum Ausdruck
               gebracht (stark—schwach; innen—außen usw.: „Gegensinn der Urworte“).}
    
         
            
            
            
        \pstart
        oder nachlässiger gearbeitet, sie haben sich an den ihnen vorliegenden Text mehr oder minder sorgfältig gehalten, die Hilfsmittel der Traumarbeit mehr oder weniger weit in Anspruch
               genommen. Im letzteren Falle erscheinen sie dunkel, verworren,
               unzusammenhängend. Wo der Traum aber greifbar absurd erscheint, einen offenbaren
               Widersinn in seinem Inhalt einschließt, da ist er so mit Absicht und bringt
               durch seine scheinbare Vernachlässigung aller logischen Anforderungen
               ein Stück vom intellektuellen Inhalt der Traumgedanken zum Ausdruck.
               Absurdität im Traum bedeutet Widerspruch, Spott und Hohn in
               den Traumgedanken. Da diese Aufklärung den stärksten Einwand gegen die
               Auffassung liefert, die den Traum durch dissoziierte, kritiklose
               Geistestätigkeit entstehen läßt, werde ich sie durch ein Beispiel zu Nachdruck
               bringen.
        \pend
    
            
        \pstart
        „Einer meiner Bekannten, Herr M., ist von keinem Geringeren als
                  von Goethe in einem Aufsatze angegriffen worden, wie wir alle meinen, mit
                  ungerechtfertigt großer Heftigkeit. — Herr M. ist durch diesen Angriff natürlich
                  vernichtet. Er beklagt sich darüber bitter bei einer Tischgesellschaft; seine
                  Verehrung für Goethe hat aber unter dieser persönlichen Erfahrung nicht gelitten.
                  Ich suche nun die zeitlichen Verhältnisse, die mir unwahrscheinlich vorkommen, ein
                  wenig aufzuklären. Goethe ist 1832 gestorben. Da sein Angriff auf Herrn M.
                  natürlich früher erfolgt sein muß, so war Herr M. damals ein ganz junger Mann. Es
                  kommt mir plausibel vor, daß er 18 Jahre alt war. Ich weiß aber nicht sicher,
                  welches Jahr wir gegenwärtig schreiben, und so versinkt die ganze Berechnung im
                  Dunkel. Der Angriff ist übrigens in dem bekannten Aufsatz von Goethe ,Natur‘
                  enthalten.“
        \pend
    
            
        \pstart
        Der Unsinn dieses Traumes tritt noch greller hervor, wenn ich mitteile, daß Herr
               M. ein jugendlicher Geschäftsmann ist, dem alle poetischen und literarischen
               Interessen ferne liegen. Wenn ich aber in die Analyse dieses Traumes eingehe,
               wird es
        \pend
    
         
            
            
            
        \pstart
        mir wohl gelingen, zu zeigen, wieviel „Methode“ hinter diesem Unsinn
               steckt. Der Traum bezieht sein Material aus drei Quellen:
        \pend
    
            
        \pstart
        1) Herr M., den ich bei einer Tischgesellschaft kennen lernte, bat mich eines Tages, seinen älteren
               Bruder zu untersuchen, der Anzeichen von gestörter geistiger Tätigkeit erkennen lasse. Bei der Unterhaltung mit dem Kranken ereignete sich das
               Peinliche, daß dieser ohne jeden Anlaß den Bruder durch eine Anspielung auf
               dessen Jugendstreiche bloßstellte. Ich hatte den Kranken um sein
               Geburtsjahr gefragt (Sterbejahr im Traum) und ihn zu verschiedenen Berechnungen veranlaßt,
               durch welche seine Gedächtnisschwäche erwiesen werden sollte.
        \pend
    
            
        \pstart
        2) Eine medizinische Zeitschrift, die sich auch meines
               Namens auf ihrem Titel rühmte, hatte von einem recht jugendlichen Referenten eine
               geradezu „vernichtende“ Kritik über ein Buch
               meines Freundes F. in Berlin aufgenommen. Ich stellte den Redakteur
               darob zur Rede, der mir zwar sein Bedauern ausdrückte, aber eine Remedur
               nicht versprechen wollte. Daraufhin brach ich meine Beziehungen zur
               Zeitung ab und hob in meinem Absagebrief die Erwartung hervor, daß unsere persönlichen Beziehungen unter diesem Vorfall nicht leiden würden.
               Dies ist die eigentliche Quelle des Traumes. Die ablehnende Aufnahme der Schrift
               meines Freundes hatte mir einen tiefen Eindruck gemacht. Sie enthielt eine
               nach meiner Schätzung fundamentale biologische Entdeckung, die erst
               jetzt — nach vielen Jahren — den Fachgenossen zu gefallen beginnt.
        \pend
    
            
        \pstart
        3) Eine Patientin hatte mir kurz zuvor die
               Krankengeschichte ihres Bruders erzählt, der mit dem Ausrufe „Natur,
               Natur“ in Tobsucht verfallen war. Die Ärzte
               hatten gemeint, der Ausruf stamme aus der Lektüre jenes schönen
        \pend
    
         
            
            
            
        \pstart
        Aufsatzes von Goethe und deute auf die
               Überarbeitung des Erkrankten bei seinen Studien hin. Ich hatte geäußert,
               es komme mir plausibler vor, daß der Ausruf
               „Natur“ in jenem sexuellen Sinn zu nehmen sei, den bei uns auch die
               Mindergebildeten kennen. Daß der Unglückliche sich später an den
               Genitalien verstümmelte, schien mir wenigstens nicht unrecht zu geben. 18 Jahre war das Alter dieses Kranken, als jener
               Anfall sich einstellte.
        \pend
    
            
        \pstart
        Im Trauminhalt verbirgt sich hinter dem Ich zunächst mein von der Kritik so übel
               behandelter Freund. „Ich suche mir die zeitlichen
                  Verhältnisse ein wenig aufzuklären.“ Das Buch meines Freundes
               beschäftigt sich nämlich mit den
               zeitlichen Verhältnissen des Lebens und führt unter
               anderem auch Goethes Lebensdauer auf ein
               Vielfaches einer für die Biologie bedeutsamen Zahl von Tagen zurück. Dieses Ich
               wird aber einem Paralytiker gleichgestellt („Ich weiß
                  nicht sicher, welches Jahr wir gegenwärtig schreiben“). Der Traum stellt
               also dar, daß mein Freund sich als Paralytiker benimmt, und schwelgt
               dabei in Absurdität. Die Traumgedanken aber lauten ironisch: „Natürlich, er ist
               ein Verrückter, ein Narr, und ihr seid die Genies, die es besser verstehen.
               Sollte es nicht doch umgekehrt sein?“ — Diese
                  Umkehrung ist nun ausgiebig im Trauminhalt
               vertreten, indem Goethe den jungen Mann
               angegriffen hat, was absurd ist, während leicht ein ganz junger Mensch noch
               heute den großen Goethe angreifen könnte.
        \pend
    
            
        \pstart
        Ich möchte behaupten, daß kein Traum von anderen als egoistischen Regungen
               eingegeben wird. Das Ich im Traum steht wirklich nicht bloß für meinen Freund,
               sondern auch für mich selbst. Ich identifiziere mich mit ihm, weil das Schicksal
               seiner Entdeckung mir vorbildlich für die Aufnahme meiner eigenen Funde erscheint. Wenn ich mit meiner
               Theorie hervortreten werde, welche in der Ätiologie psychoneurotischer Störungen
               die Sexualität hervorhebt (siehe die Anspielung auf den achtzehn-
        \pend
    
         
            
            
            
        \pstart
        jährigen Kranken „Natur, Natur“), werde
               ich die nämliche Kritik wiederfinden und bringe ihr schon jetzt den gleichen
               Spott entgegen.
        \pend
    
            
        \pstart
        Wenn ich die Traumgedanken weiter verfolge, finde ich immer nur Spott und Hohn als das Korrelat der Absurditäten des
                  Traumes. Der Fund eines geborstenen Schafschädels auf dem Lido zu
               Venedig hat Goethe bekanntlich die Idee zur sog.
               Wirbeltheorie des Schädels eingegeben. — Mein Freund rühmt sich, als Student
               einen Sturm zur Beseitigung eines alten Professors entfesselt zu haben, der
               einst, wohlverdient (unter anderem auch um diesen Teil der vergleichenden
               Anatomie), nun durch Altersschwachsinn zum
               Lehren unfähig geworden war. Die von ihm veranstaltete Agitation half so dem
               Übelstande ab, daß an den deutschen Universitäten dem akademischen Wirken
               eine Altersgrenze nicht gezogen ist. — Alter schützt nämlich vor Torheit nicht. — Im hiesigen
               Krankenhause hatte ich die Ehre, Jahre hindurch unter einem Primarius zu
               dienen, der längst fossil, seit Dezennien notorisch
                  schwachsinnig, sein verantwortungsvolles Amt weiterführen durfte. Eine
               Charakteristik nach dem Funde am Lido drängt sich mir hier auf. — Auf diesen
               Mann bezüglich fertigten einst junge Kollegen im Spital eine Übertragung des
               damals beliebten Gassenhauers: Das hat kein Goethe g’schrieben, das hat kein Schiller g’dicht usw. . . .
        \pend
    
         
            
            
            \pstart\eledsection*{VII}\pend
            
        \pstart
        Wir sind mit der Würdigung der Traumarbeit noch nicht zu Ende gekommen. Wir
               sehen uns genötigt, ihr außer der Verdichtung, Verschiebung und
               anschaulichen Zurichtung des psychischen Materials noch eine andere
               Tätigkeit zuzuschreiben, deren Beitrag allerdings nicht an allen Träumen zu
               erkennen ist. Ich werde von diesem Stück der Traumarbeit nicht ausführlich
               handeln, will also nur anführen, daß man sich von seinem Wesen am ehesten eine
               Vorstellung verschafft, wenn man sich zu der — wahrscheinlich unzutreffenden —
               Annahme entschließt, daß es auf den bereits vorgebildeten
                  Trauminhalt erst nachträglich einwirke. Seine Leistung besteht dann
               darin, die Traumbestandteile so anzuordnen, daß sie sich ungefähr zu einem
               Zusammenhang, zu einer Traumkomposition zusammenfügen. Der Traum
               erhält so eine Art Fassade, die seinen Inhalt freilich nicht an allen Stellen
               deckt; er erfährt dabei eine erste vorläufige Deutung, die durch Einschiebsel
               und leise Abänderungen unterstützt wird. Allerdings macht sich diese Bearbeitung
               des Trauminhaltes nur möglich, indem sie alle fünf gerade sein läßt, sie
               liefert auch weiter nichts als ein eklatantes Mißverständnis der Traumgedanken,
               und wenn wir die Analyse des Traumes in Angriff nehmen, müssen wir uns zuerst
               von diesem Deutungsversuche frei machen.
        \pend
    
            
        \pstart
        An diesem Stücke der Traumarbeit ist die Motivierung ganz besonders
               durchsichtig. Es ist die Rücksicht auf Ver-
        \pend
    
         
            
            
            
        \pstart
        ständlichkeit, welche diese letzte
               Überarbeitung des Traumes veranlaßt; hiedurch ist aber auch die Herkunft dieser
               Tätigkeit verraten. Sie benimmt sich gegen den ihr vorliegenden Trauminhalt, wie unsere normale psychische Tätigkeit überhaupt gegen
               einen beliebigen ihr dargebotenen Wahrnehmungsinhalt. Sie erfaßt ihn unter
               Verwendung gewisser Erwartungsvorstellungen, ordnet ihn schon bei der
               Wahrnehmung unter der Voraussetzung seiner Verständlichkeit, läuft dabei Gefahr,
               ihn zu fälschen, und verfällt in der Tat, wenn er sich an nichts Bekanntes
               anreihen läßt, zunächst in die seltsamsten Mißverständnisse. Es ist
               bekannt, daß wir nicht imstande sind, eine Reihe von fremdartigen Zeichen
               anzusehen oder ein Gefolge von unbekannten Worten anzuhören, ohne zunächst deren
               Wahrnehmung nach der Rücksicht auf Verständlichkeit,
               nach der Anlehnung an etwas uns Bekanntes zu verfälschen.
        \pend
    
            
        \pstart
        Träume, welche diese Bearbeitung von seiten einer dem wachen Denken völlig
               analogen psychischen Tätigkeit erfahren haben, kann man gut komponierte heißen. Bei anderen Träumen hat
               diese Tätigkeit völlig versagt; es ist nicht einmal der Versuch gemacht worden,
               Ordnung und Deutung herzustellen, und indem wir uns nach dem Erwachen mit diesem
               letzten Stück der Traumarbeit identisch fühlen, urteilen wir, der
               Traum sei „ganz verworren“. Für unsere Analyse aber hat der Traum,
               der einem ordnungslosen Haufen unzusammenhängender Bruchstücke gleicht,
               ebensoviel Wert wie der schön geglättete und mit einer Oberfläche
               versehene. Wir ersparen uns im ersteren Falle etwa die Mühe, die Überarbeitung
               des Trauminhaltes wieder zu zerstören.
        \pend
    
            
        \pstart
        Man würde aber irre gehen, wenn man in diesen Traumfassaden nichts
               anderes sehen wollte, als solche eigentlich mißverständliche und
               ziemlich willkürliche Bearbeitungen des Trauminhaltes durch die
               bewußte Instanz unseres Seelenlebens. Zur Herstellung der Traumfassade werden
               nicht selten Wunschphantasien verwendet, die sich in den
               Traumgedanken vorgebildet
        \pend
    
         
            
            
            
        \pstart
        finden, und die von derselben Art sind wie die uns aus dem wachen
               Leben bekannten, mit Recht so genannten „Tagträume“. Die Wunschphantasien,
               welche die Analyse in den nächtlichen Träumen aufdeckt, erweisen sich oft als
               Wiederholungen und Umarbeitungen infantiler Szenen; die Traumfassade zeigt uns
               so in manchen Träumen unmittelbar den durch Vermengung mit anderem Material
               entstellten eigentlichen Kern des Traumes.
        \pend
    
            
        \pstart
        Andere als die vier erwähnten Tätigkeiten sind bei der Traumarbeit
               nicht zu entdecken. Halten wir an der Begriffsbestimmung fest, daß „Traumarbeit“
               die Überführung der Traumgedanken in den Trauminhalt bezeichnet, so müssen wir
               uns sagen, die Traumarbeit sei nicht schöpferisch, sie entwickle
               keine ihr eigentümliche Phantasie, sie \edtext{urteilt}{\lemma{\textbf{urteilt}}\Aendnote{urtheilt E }} nicht, schließt nicht, sie leistet überhaupt nichts anderes als das
               Material zu verdichten, verschieben und auf Anschaulichkeit umzuarbeiten, wozu
               noch das inkonstante letzte Stückchen deutender Bearbeitung hinzukommt. Man
               findet zwar mancherlei im Trauminhalt, was man als das Ergebnis einer
               anderen und höheren intellektuellen Leistung auffassen möchte, aber die Analyse
               weist jedesmal überzeugend nach, daß
               diese intellektuellen Operationen bereits in den
                  Traumgedanken vorgefallen und vom Trauminhalt nur übernommen worden sind.
               Eine Schlußfolgerung im Traum ist nichts anderes als die Wiederholung eines
               Schlusses in den Traumgedanken; sie erscheint unanstößig, wenn sie ohne
               Veränderung in den Traum übergegangen ist; sie wird unsinnig, wenn sie durch die
               Traumarbeit etwa auf ein anderes Material verschoben wurde. Eine Rechnung im
               Trauminhalt bedeutet nichts anderes, als daß sich unter den Traumgedanken eine
               Berechnung findet; während diese jedesmal richtig ist, kann die Traumrechnung durch Verdichtung ihrer Faktoren und durch Verschiebung der nämlichen Operationsweise auf anderes Material das tollste
               Ergebnis liefern. Nicht einmal die Reden, die sich im Trauminhalt vorfinden,
               sind neu komponiert; sie erweisen
        \pend
    
         
            
            
            
        \pstart
        sich als zusammengestückelt aus Reden, die als gehaltene oder als
               gehörte und gelesene in den Traumgedanken erneuert wurden, deren Wortlaut sie
               aufs getreueste kopieren, während sie deren Veranlassung ganz beiseite lassen
               und ihren Sinn aufs gewaltsamste verändern.
        \pend
    
            
        \pstart
        Es ist vielleicht nicht überflüssig, die letzten Behauptungen durch Beispiele zu
               unterstützen.
        \pend
    
            
        \pstart
        I) Ein harmlos klingender, gut komponierter Traum einer
               Patientin:
        \pend
    
            
        \pstart
        Sie geht auf den Markt mit ihrer Köchin, die den Korb trägt. Der
                  Fleischhauer sagt ihr, nachdem sie etwas verlangt hat: Das ist nicht mehr zu
                  haben, und will ihr etwas anderes geben mit der Bemerkung: Das ist auch gut. Sie
                  lehnt ab und geht zur Gemüsefrau. Die will ihr ein eigentümliches Gemüse
                  verkaufen, was in Bündeln zusammengebunden ist, aber schwarz von Farbe. Sie sagt:
                  Das kenne ich nicht, das nehme ich nicht.
        \pend
    
            
        \pstart
        Die Rede: das ist nicht mehr zu haben — stammt aus der Behandlung. Ich selbst
               hatte der Patientin einige Tage vorher wörtlich erklärt, daß die ältesten
               Kindererinnerungen nicht mehr als solche zu haben sind, sondern sich durch Übertragungen und Träume ersetzen. Ich bin also der Fleischhauer.
        \pend
    
            
        \pstart
        Die zweite Rede: Das kenne ich nicht — ist in
               einem ganz anderen Zusammenhange vorgefallen. Tags vorher hatte sie selbst
               ihrer Köchin, die übrigens auch im Traume erscheint, tadelnd zugerufen: Benehmen Sie sich anständig; das kenne ich nicht, d.
               h. wohl, ein solches Benehmen anerkenne ich nicht, lasse ich nicht zu. Der
               harmlosere Teil dieser Rede gelangte durch eine Verschiebung in den Trauminhalt;
               in den Traumgedanken spielte nur der andere Teil der Rede eine Rolle, denn
               hier hat die Traumarbeit bis zur vollen Unkenntlichkeit und bis zur äußersten
               Harmlosigkeit eine Phantasiesituation verändert, in welcher ich mich
               gegen die Dame in einer gewissen Weise unanständig
                  benehme. Diese in
        \pend
    
         
            
            
            
        \pstart
        der Phantasie erwartete Situation ist aber selbst nur die Neuauflage einer einmal wirklich erlebten.
        \pend
    
            
        \pstart
        II) Ein scheinbar ganz bedeutungsloser Traum, in dem
               Zahlen vorkommen. Sie will irgend etwas bezahlen; ihre
                  Tochter nimmt 3 fl. 65 kr. aus der Geldtasche; sie sagt aber: Was tust du? Es
                  kostet ja nur 21 Kreuzer.
        \pend
    
            
        \pstart
        Die Träumerin war eine Fremde, die ihr Kind in einem Wiener Erziehungsinstitute
               untergebracht hatte, und die meine Behandlung fortsetzen konnte, so lange ihre
               Tochter in Wien blieb. Am Tage vor dem Traume hatte ihr die Institutsvorsteherin nahegelegt, ihr das Kind noch ein weiteres Jahr zu
               überlassen. In diesem Falle hätte sie auch die Behandlung um ein Jahr
               verlängert. Die Zahlen im Traum kommen zur Bedeutung, wenn man sich
               erinnert, daß Zeit Geld ist. Time is money.
               Ein Jahr ist gleich 365 Tagen, in Kreuzern ausgedrückt 365 Kreuzer oder 3 l. 65 kr. Die 21 Kreuzer entsprechen den
               drei Wochen, die damals vom Traumtage bis zum
               Schulschluß und damit bis zum Ende der Kur ausständig waren. Es waren
               offenbar Geldrücksichten, welche die Dame bewogen hatten, den Vorschlag der
               Vorsteherin abzulehnen, und welche für die Kleinheit der Summe im
               Traum verantwortlich sind.
        \pend
    
            
        \pstart
        III. Eine junge, aber schon seit Jahren verheiratete Dame erfährt, daß eine ihr
               fast gleichalterige Bekannte, Frl. Elise L., sich verlobt hat. Dieser Anlaß
               erregt nachstehenden Traum:
        \pend
    
            
        \pstart
        Sie sitzt mit ihrem Manne im Theater, eine Seite des Parketts ist
                  ganz unbesetzt. Ihr Mann erzählt ihr, Elise L. und ihr Bräutigam hätten auch gehen
                  wollen, hätten aber nur schlechte Sitze bekommen, drei für 1 fl. 50 kr., und die
                  konnten sie ja nicht nehmen. Sie meint, es wäre auch kein Unglück
               gewesen.
        \pend
    
            
        \pstart
        Hier wird uns die Herkunft der Zahlen aus dem Material der Traumgedanken und die
               Verwandlungen, die sie erfahren haben, interessieren. Woher rühren die 1 fl. 50 kr.? Aus einem indiffe-
        \pend
    
         
            
            
            
        \pstart
        renten Anlaß des Vortages. Ihre Schwägerin hatte von ihrem Manne die
               Summe von 150 fl. zum Geschenke bekommen und sich beeilt, sie los zu werden, indem sie sich einen Schmuck dafür kaufte.
               Wir wollen anmerken, daß 150 fl. hundertmal mehr sind als 1 fl. 50 kr. Für die
                  drei, die bei den Theaterbillets steht,
               findet sich nur die eine Anknüpfung, daß die Braut Elise L. genau drei Monate jünger ist als die Träumerin. Die Situation im Traume ist die Wiedergabe einer kleinen Begebenheit, mit
               der sie von ihrem Manne oft geneckt worden ist. Sie hatte sich einmal so sehr
                  beeilt, vorzeitig Karten zu einer Theatervorstellung zu nehmen, und als sie dann ins Theater kam, war eine Seite des Parketts fast unbesetzt. Sie hätte
               es also nicht nötig gehabt, sich so sehr zu
                  beeilen. — Übersehen wir endlich nicht die Absurdität des Traumes, daß zwei Personen drei
               Karten fürs Theater nehmen sollen!
        \pend
    
            
        \pstart
        Nun die Traumgedanken: Ein Unsinn war es doch, so
               früh zu heiraten; ich hätte es nicht nötig gehabt,
                  mich so zu beeilen. An dem Beispiel der Elise L. sehe ich, daß ich immer
               noch einen Mann bekommen hätte, und zwar einen hundertmal besseren (Mann,
               Schatz), wenn ich nur gewartet hätte.
               Drei solche Männer hätte ich mir für das Geld (die
               Mitgift) kaufen können!
        \pend
    
         
            
            
            \pstart\eledsection*{VIII}\pend
            
        \pstart
        Nachdem wir in den vorstehenden Darlegungen die Traumarbeit kennen
               gelernt haben, werden wir wohl geneigt sein, sie für einen ganz besonderen
               psychischen Vorgang zu erklären, dessengleichen es nach unserer Kenntnis sonst
               nicht gibt. Es ist gleichsam auf die Traumarbeit das Befremden
               übergegangen, welches sonst ihr Produkt, der Traum, bei uns zu erwecken
               pflegte. In Wirklichkeit ist die Traumarbeit nur der zuerst erkannte unter einer
               ganzen Reihe von psychischen Prozessen, auf welche die Entstehung der
               hysterischen Symptome, der Angst-, Zwangs- und Wahnideen zurückzuführen ist.
               Verdichtung und vor allem Verschiebung sind niemals fehlende Charaktere
               auch dieser anderen Prozesse. Die Umarbeitung aufs Anschauliche bleibt
               hingegen der Traumarbeit eigentümlich. Wenn diese Aufklärung den Traum in eine
               Reihe mit den Bildungen psychischer Erkrankung bringt, so wird es uns
               um so wichtiger werden, die wesentlichen Bedingungen solcher Vorgänge wie der
               Traumbildung zu erfahren. Wir werden wahrscheinlich verwundert sein zu
               hören, daß weder Schlafzustand noch Krankheit zu diesen unentbehrlichen Bedingungen gehören. Eine ganze Anzahl von Phänomenen des
               Alltagslebens Gesunder, das Vergessen, Versprechen, Vergreifen, und
               eine gewisse Klasse von Irrtümern danken einem analogen psychischen Mechanismus
               wie der Traum und die anderen Glieder der Reihe ihre Entstehung.
        \pend
    
            
        \pstart
        Der Kern des Problems liegt in der Verschiebung, der weitaus
        \pend
    
         
            
            
            
        \pstart
        auffälligsten unter den Einzelleistungen der Traumarbeit. Die
               wesentliche Bedingung der Verschiebung lernt man bei eingehender
               Vertiefung in den Gegenstand als eine rein psychologische kennen; sie ist von
               der Art einer Motivierung. Man gerät auf ihre
               Spur, wenn man Erfahrungen würdigt, denen man bei der Analyse von Träumen nicht
               entgehen kann. Ich habe bei der Analyse des Traumbeispiels auf Seite 198 in der
               Mitteilung der Traumgedanken abbrechen müssen, weil sich unter ihnen, wie
               ich eingestand, solche fanden, die ich gerne vor Fremden geheim halte und ohne
               schwere Verletzung wichtiger Rücksichten nicht mitteilen kann. Ich fügte hinzu,
               es brächte gar keinen Nutzen, wenn ich anstatt dieses Traumes einen anderen zur
                  Mitteilung seiner Analyse auswählte; bei jedem Traum, dessen
               Inhalt dunkel oder verworren ist, würde ich auf Traumgedanken stoßen, die
               Geheimhaltung erfordern. Wenn ich aber für mich selbst die Analyse fortsetze,
               ohne Rücksicht auf die anderen, für die ja ein so persönliches Erlebnis wie mein
               Traum gar nicht bestimmt sein kann, so lange ich endlich bei Gedanken an,
               die mich überraschen, die ich in mir nicht gekannt habe, die mir aber nicht
                  nur fremdartig, sondern auch unangenehm sind, und die ich darum energisch bestreiten möchte, während die
               durch die Analyse laufende Gedankenverkettung sie mir unerbittlich aufdrängt.
               Ich kann diesem ganz allgemeinen Sachverhalt gar nicht anders Rechnung tragen,
               als durch die Annahme, diese Gedanken seien wirklich in meinem Seelenleben
               vorhanden und im Besitz einer gewissen psychischen Intensität oder Energie
               gewesen, hätten sich aber in einer eigentümlichen psychologischen Situation
               befunden, der zufolge sie mir nicht bewußt werden
               konnten. Ich heiße diesen besonderen Zustand den der Verdrängung. Ich kann dann nicht umhin, zwischen der Dunkelheit des
               Trauminhaltes und dem Verdrängungszustand, der Bewußtseinsunfähigkeit, einiger der Traumgedanken eine kausale
               Beziehung gelten zu lassen und zu schließen, daß
        \pend
    
         
            
            
            
        \pstart
        der Traum dunkel sein müsse, damit er die
                  verpönten Traumgedanken nicht verrate. Ich komme so zum Begriffe der Traumentstellung, welche das Werk der Traumarbeit
               ist, und der Verstellung, der Absicht zu
               verbergen, dient.
        \pend
    
            
        \pstart
        Ich will an dem zur Analyse ausgesuchten Traumbeispiel die Probe machen und mich
               fragen, welches denn der Gedanke ist, der sich in diesem Traum entstellt zur
               Geltung bringt, während er unentstellt meinen schärfsten Widerspruch
               herausfordern würde. Ich erinnere mich, daß die kostenlose Wagenfahrt mich an
               die letzten kostspieligen Wagenfahrten mit einer Person meiner Familie
               gemahnt hat, daß sich als die Deutung des Traumes ergab: Ich möchte einmal Liebe
               kennen lernen, die mich nichts kostet, und daß ich kurze Zeit vor dem Traum eine
               größere Geldausgabe für eben diese Person zu leisten hatte. In diesem
               Zusammenhang kann ich mich des Gedankens nicht erwehren,
               daß es mir um diese Ausgabe leid tut. Erst wenn
               ich diese Regung anerkenne, bekommt es einen Sinn, daß ich mir im Traum
               Liebe wünsche, die mir keine Ausgabe nötig macht. Und doch kann ich mir ehrlich
               sagen, daß ich bei der Entschließung, jene Summe aufzuwenden, nicht
               einen Augenblick geschwankt habe. Das Bedauern darüber, die Gegenströmung,
               ist mir nicht bewußt worden. Aus welchen Gründen nicht, dies ist allerdings
               eine andere, weitab führende Frage, deren mir bekannte Beantwortung in einen
               anderen Zusammenhang gehört.
        \pend
    
            
        \pstart
        Wenn ich nicht einen eigenen Traum, sondern den einer fremden Person der Analyse
               unterziehe, so ist das Ergebnis das nämliche; die Motive zur Überzeugung werden
               aber geändert. Ist es der Traum eines Gesunden, so bleibt mir kein anderes
               Mittel, ihn zur Anerkennung der gefundenen verdrängten Ideen zu nötigen, als der
               Zusammenhang der Traumgedanken, und er mag sich immerhin gegen diese Anerkennung
               sträuben. Handelt es sich aber um einen neurotisch Leidenden, etwa um einen
               Hysteriker, so wird die Annahme des verdrängten Gedankens
        \pend
    
         
            
            
            
        \pstart
        für ihn zwingend durch den Zusammenhang dieses letzteren mit seinen
               Krankheitssymptomen und durch die Besserung, die er bei dem Eintausch von
               Symptomen gegen verdrängte Ideen erfährt. Bei der Patientin z. B., von welcher
               der letzte Traum mit den drei Karten für 1 fl. 50 kr. herrührt, muß die
               Analyse annehmen, daß sie ihren Mann geringschätzt, daß sie bedauert, ihn
               geheiratet zu haben, daß sie ihn gerne gegen einen anderen vertauschen möchte.
               Sie behauptet freilich, daß sie ihren Mann liebt, daß ihr Empfindungsleben von
               dieser Geringschätzung (einen hundertmal besseren!) nichts weiß, aber
               all ihre Symptome führen zu derselben Auflösung wie dieser Traum, und nachdem
               die von ihr verdrängten Erinnerungen an eine gewisse Zeit wieder geweckt
               worden sind, in welcher sie ihren Mann auch bewußt nicht geliebt hat, sind diese
               Symptome gelöst, und ihr Widerstand gegen die Deutung des Traumes ist
               geschwunden.
        \pend
    
         
            
            
            \pstart\eledsection*{IX}\pend
            
        \pstart
        Nachdem wir uns den Begriff der Verdrängung fixiert und die Traumentstellung in
               Beziehung zu verdrängtem psychischen Material gesetzt haben, können wir das
               Hauptergebnis, welches die Analyse der Träume liefert, ganz allgemein
               aussprechen. Von den verständlichen und sinnvollen Träumen haben wir
               erfahren, daß sie unverhüllte Wunscherfüllungen sind, d. h. daß die Traumsituation in ihnen einen dem Bewußtsein bekannten, vom Tagesleben erübrigten, des Interesses wohl würdigen Wunsch als erfüllt
               darstellt. Über die dunkeln und verworrenen Träume lehrt nun die Analyse etwas
               ganz Analoges: die Traumsituation stellt wiederum einen Wunsch als erfüllt dar,
               der sich regelmäßig aus den Traumgedanken erhebt, aber die
               Darstellung ist eine unkenntliche, erst durch Zurückführung in der Analyse
               aufzuklärende, und der Wunsch ist entweder selbst ein verdrängter,
               dem Bewußtsein fremder, oder er hängt doch innigst mit verdrängten Gedanken
               zusammen, wird von solchen getragen. Die Formel für diese Träume lautet also:
                  Sie sind verhüllte Erfüllungen von verdrängten Wünschen. Es
               ist dabei interessant zu bemerken, daß die Volksmeinung recht behält, welche den
               Traum durchaus die Zukunft verkünden läßt. In Wahrheit ist die Zukunft, die uns
               der Traum zeigt, nicht die, die eintreffen wird, sondern von der wir
               möchten, daß sie so einträfe. Die Volksseele verfährt hier, wie sie es auch
               sonst gewohnt ist: sie glaubt, was sie wünscht.
        \pend
    
         
            
            
            
        \pstart
        Nach ihrem Verhalten gegen die Wunscherfüllung teilen sich die Träume in drei
               Klassen. Erstens solche, die einen unverdrängten Wunsch unverhüllt darstellen; dies sind die Träume von
               infantilem Typus, die beim Erwachsenen immer seltener werden. Zweitens die
               Träume, die einen verdrängten Wunsch verhüllt zum Ausdruck bringen; wohl die übergroße
               Mehrzahl aller unserer Träume, die zum Verständnis dann der Analyse bedürfen.
               Drittens die Träume, die zwar einen verdrängten Wunsch darstellen, aber ohne oder in ungenügender Verhüllung.
               Diese letzten Träume sind regelmäßig von
               Angst begleitet, welche den Traum unterbricht. Die
               Angst ist hier der Ersatz für die Traumentstellung; sie ist mir in den
               Träumen der zweiten Klasse durch die Traumarbeit erspart worden. Es läßt sich
               ohne allzugroße Schwierigkeit nachweisen, daß derjenige Vorstellungsinhalt, der
               uns jetzt im Traume Angst bereitet, einstmals ein Wunsch war und seither der
               Verdrängung unterlegen ist.
        \pend
    
            
        \pstart
        Es gibt auch klare Träume von peinlichem Inhalt, der aber im Traum nicht
               peinlich empfunden wird. Man kann diese darum nicht zu den Angstträumen rechnen;
               sie haben aber immer dazu gedient, die Bedeutungslosigkeit und den psychischen
               Unwert der Träume zu erweisen. Eine Analyse eines solchen Beispieles wird
               zeigen, daß es sich hier um gut verhüllte Erfüllungen
                  verdrängter Wünsche, also um Träume der zweiten Klasse,
               handelt und wird gleichzeitig die ausgezeichnete Eignung der Verschiebungsarbeit zur Verhüllung des Wunsches dartun.
        \pend
    
            
        \pstart
        Ein Mädchen träumt, daß sie das jetzt einzige Kind ihrer Schwester tot vor sich
               sieht in der nämlichen Umgebung, in der sie vor einigen Jahren das erste Kind
               als Leiche sah. Sie empfindet dabei keinen Schmerz, sträubt sich aber
               natürlich gegen die Auffassung, diese Situation entspreche einem Wunsche
               von ihr. Dies wird auch nicht erfordert; aber an der Bahre jenes Kindes hat
               sie vor Jahren den von ihr geliebten Mann zuletzt
        \pend
    
         
            
            
            
        \pstart
        gesehen und gesprochen; stürbe das zweite Kind, so würde sie diesen
               Mann gewiß wieder im Hause der Schwester treffen. Sie sehnt sich nun nach dieser
               Begegnung, sträubt sich aber gegen dieses ihr Gefühl. Sie hat am Traumtage
               selbst eine Eintrittskarte zu einem Vortrage genommen, den der immer noch
               Geliebte angekündigt hat. Ihr Traum ist ein einfacher Ungeduldstraum,
               wie er sich gewöhnlich vor Reisen, Theaterbesuchen und ähnlichen erwarteten
               Genüssen einstellt. Um ihr aber diese Sehnsucht zu verbergen, ist die Situation
               auf die für eine freudige Empfindung unpassendste Gelegenheit verschoben worden,
               die sich doch einmal in der Wirklichkeit bewährt hat. Man beachte
               noch, daß das Affektverhalten im Traume nicht dem vorgeschobenen, sondern
               dem wirklichen, aber zurückgehaltenen Trauminhalt angepaßt ist. Die
               Traumsituation greift dem lange ersehnten Wiedersehen vor; sie bietet keine
               Anknüpfung für eine schmerzliche Empfindung.
        \pend
    
         
            
            
            \pstart\eledsection*{X}\pend
            
        \pstart
        Die Philosophen haben bisher keinen Anlaß gehabt, sich mit einer Psychologie der
               Verdrängung zu beschäftigen. Es ist also gestattet, daß wir uns in erster
               Annäherung an den noch unbekannten Sachverhalt eine anschauliche
               Vorstellung vom Hergang der Traumbildung schaffen. Das Schema, zu welchem wir
               nicht allein vom Studium des Traumes her gelangen, ist zwar bereits
               ziemlich kompliziert; wir können aber mit einem einfacheren unser Ausreichen
               nicht finden. Wir nehmen an, daß es in unserem seelischen Apparat zwei
               gedankenbildende Instanzen gibt, deren zweite das Vorrecht besitzt, daß ihre
               Erzeugnisse den Zugang zum Bewußtsein offen finden, während die Tätigkeit der
               ersten Instanz an sich unbewußt ist und nur über die zweite zum Bewußtsein
               gelangen kann. An der Grenze der beiden Instanzen, am Übergang von
               der ersten zur zweiten, befinde sich eine Zensur, welche nur durchläßt, was ihr
               angenehm ist, anderes aber zurückhält. Dann befindet sich das von der Zensur
               Abgewiesene, nach unserer Definition, im Zustande der Verdrängung. Unter
               gewissen Bedingungen, deren eine der Schlafzustand ist, ändere sich
               das Kräfteverhältnis zwischen beiden Instanzen in solcher Weise, daß
               das Verdrängte nicht mehr ganz zurückgehalten werden kann. Im Schlafzustand
               geschehe dies etwa durch den Nachlaß der Zensur; dann wird es dem bisher
               Verdrängten gelingen, sich den Weg zum Bewußtsein zu bahnen. Da die Zensur aber
               niemals aufgehoben, sondern bloß herabgesetzt ist, so wird es sich
               dabei Ver-
        \pend
    
         
            
            
            
        \pstart
        änderungen gefallen lassen müssen, welche seine Anstößigkeiten
               mildern. Was in solchem Falle bewußt wird, ist ein Kompromiß zwischen dem von
               der einen Instanz Beabsichtigten und dem von der anderen Geforderten. Verdrängung — Nachlaß der Zensur — Kompromißbildung,
               dies ist aber das Grundschema für die Entstehung sehr vieler anderer
               psychopathischer Bildungen in gleicher Weise wie für den Traum, und bei der
               Kompromißbildung werden hier wie dort die Vorgänge der Verdichtung
               und Verschiebung und die Inanspruchnahme oberflächlicher
               Assoziationen beobachtet, welche wir bei der Traumarbeit kennen gelernt
               haben.
        \pend
    
            
        \pstart
        Wir haben keinen Grund, uns das Element von Dämonismus zu verhehlen, welches bei
               der Aufstellung unserer Erklärung der Traumarbeit mitgespielt hat. Wir haben den
               Eindruck empfangen, daß die Bildung der dunklen Träume so vor sich geht, als ob eine Person, die von einer zweiten
               abhängig ist, etwas zu äußern hätte, was dieser letzteren anzuhören unangenehm
               sein muß, und von diesem Gleichnis her haben wir den Begriff der Traumentstellung und den der Zensur erfaßt und uns bemüht, unseren Eindruck
               in eine gewiß rohe, aber wenigstens anschauliche psychologische
               Theorie zu übersetzen. Mit was immer bei weiterer Klärung des Gegenstandes sich
               unsere erste und zweite Instanz wird identifizieren lassen, wir werden erwarten,
               daß sich ein Korrelat unserer Annahme bestätige, daß die zweite Instanz den
               Zugang zum Bewußtsein beherrscht und die erste vom Bewußtsein absperren
               kann.
        \pend
    
            
        \pstart
        Wenn der Schlafzustand überwunden ist, stellt sich die Zensur rasch zur vollen
               Höhe wieder her und kann jetzt wieder vernichten, was ihr während der
               Zeit ihrer Schwäche abgerungen worden ist. Daß das Vergessen des Traumes wenigstens zum Teil
               diese Erklärung fordert, geht aus einer ungezählte Male bestätigten Erfahrung
               hervor. Während der Erzählung eines Traumes oder während der Analyse desselben
               geschieht es
            
        \pend
    
         
            
            
            
        \pstart
        nicht selten, daß plötzlich ein vergessen geglaubtes Bruchstück des
               Trauminhaltes wieder auftaucht. Dies dem Vergessen entrissene Stück
               enthält regelmäßig den besten und nächsten Zugang zur Bedeutung des Traumes. Es
               sollte wahrscheinlich nur darum dem Vergessen, d. i. der neuerlichen
               Unterdrückung, verfallen.
            
        \pend
    
         
            
            
            \pstart\eledsection*{XI}\pend
            
        \pstart
        Wenn wir den Trauminhalt als Darstellung eines erfüllten Wunsches auffassen und
               seine Dunkelheit auf die Abänderungen der Zensur an verdrängtem Material
               zurückführen, fällt es uns auch nicht mehr schwer, die Funktion des Traumes zu
               erschließen. In seltsamem Gegensatz zu Redewendungen, welche den Schlaf
               durch Träume stören lassen, müssen wir den Traum als den
                  Hüter des Schlafes anerkennen. Für den Kindertraum dürfte unsere
               Behauptung leicht Glauben finden.
        \pend
    
            
        \pstart
        Der Schlafzustand oder die psychische Schlafveränderung, worin immer sie
               bestehen mag, wird herbeigeführt durch den dem Kind aufgenötigten oder auf Grund
               von Müdigkeitssensationen gefaßten Entschluß zu schlafen, und einzig ermöglicht
               durch die Abhaltung von Reizen, welche dem psychischen Apparat andere Ziele
               setzen könnten als das des Schlafens. Die Mittel, welche dazu dienen, äußere
               Reize ferne zu halten, sind bekannt; aber welche Mittel stehen uns zur
               Verfügung, um die inneren seelischen Reize niederzuhalten, die sich
               dem Einschlafen widersetzen? Man beobachte eine Mutter, die ihr Kind
               einschläfert. Es äußert unausgesetzt Bedürfnisse, es will noch einen Kuß, es
               möchte noch spielen. Diese Bedürfnisse werden zum Teil befriedigt, zum
               anderen mit Autorität auf den nächsten Tag verschoben. Es ist klar, daß
               Wünsche und Bedürfnisse, die sich regen, die Hemmnisse des Einschlafens sind.
               Wer kennt nicht die heitere Geschichte von dem schlimmen Buben (Balduin Grollers),
                  \edtext{}{\Bendnote{
                  \textbf{die heitere Geschichte von dem schlimmen Buben (Balduin Grollers)}]
                     Vom kleinen Rudi (1897), Roman von
                  Balduin Groller (d.i. Adalbert Goldscheider), dem Erfinder des österreichischen
                  Sherlock Holmes, alias Detektiv Dagobert Trostler. }}
     der, bei Nacht
        \pend
    
         
            
            
            
        \pstart
        erwachend, durch den Schlafraum brüllt: Das
                  Nashorn will er? Ein braveres Kind würde, anstatt zu brüllen, träumen, daß es mit dem Nashorn spiele. Da der
               Traum, welcher den Wunsch erfüllt zeigt, während des Schlafens Glauben findet, hebt er den Wunsch auf und
               ermöglicht den Schlaf. Es ist nicht abzuweisen, daß dieser Glaube dem
               Traumbilde zufällt, weil dieses sich in die psychische Erscheinung der
               Wahrnehmung kleidet, während dem Kinde die später zu erwerbende Fähigkeit
               noch fehlt, Halluzination oder Phantasie von Realität zu unterscheiden.
        \pend
    
            
        \pstart
        Der Erwachsene hat diese Unterscheidung gelernt, er hat auch die Nutzlosigkeit
               des Wünschens begriffen und durch fortgesetzte Übung erreicht, seine Strebungen
               aufzuschieben, bis sie auf langen Umwegen über die Veränderung der Außenwelt
               ihre Erledigung finden können. Dem entsprechend sind auch die
               Wunscherfüllungen auf kurzem psychischen Weg bei ihm im Schlafe selten; ja,
               es ist selbst möglich, daß sie überhaupt nicht vorkommen, und daß alles,
               was uns nach der Art eines Kindertraumes gebildet zu sein scheint, eine viel
               kompliziertere Auflösung erfordert. Dafür aber hat sich beim Erwachsenen — und
               wohl bei jedem Vollsinnigen ohne Ausnahme — eine Differenzierung des psychischen
               Materiales herausgebildet, die dem Kinde fehlte. Es ist eine psychische
               Instanz zustande gekommen, welche, durch die Lebenserfahrung belehrt, einen
               beherrschenden und hemmenden Einfluß auf die seelischen Regungen mit
               eifersüchtiger Strenge festhält, und die durch ihre Stellung zum Bewußtsein und
               zur willkürlichen Motilität mit den größten Mitteln psychischer Macht
               ausgestattet ist. Ein KeilTeil der kindlichen Regungen aber ist als
               lebensunnütz von dieser Instanz unterdrückt worden, und alles Gedankenmaterial, was von diesen abstammt, befindet sich im Zustande der
               Verdrängung.
        \pend
    
            
        \pstart
        Während sich nun die Instanz, in welcher wir unser normales Ich erkennen, auf
               den Wunsch zu schlafen, einstellt, scheint sie durch die psychophysiologischen
               Bedingungen des Schlafes genötigt,
        \pend
    
         
            
            
            
        \pstart
        an der Energie nachzulassen, mit welcher sie bei Tag das Verdrängte niederzuhalten pflegte. Dieser Nachlaß selbst ist zwar harmlos; die Erregungen der unterdrückten Kinderseele mögen sich
               immerhin tummeln; infolge des nämlichen Schlafzustandes finden sie doch den
               Zugang zum Bewußtsein erschwert und den zur Motilität versperrt. Die Gefahr, daß
               der Schlaf durch sie gestört werde, muß aber abgewehrt werden. Nun müssen wir ja
                  ohnehin die Annahme zulassen, daß selbst im tiefen Schlaf ein
               Betrag von freier Aufmerksamkeit als Wächter gegen Sinnesreize aufgeboten wird, welche etwa das Erwachen rätlicher erscheinen lassen
               als die Fortsetzung des Schlafes. Es wäre sonst nicht zu erklären, daß wir
               jederzeit durch Sinnesreize von gewisser Qualität
                  aufzuwecken sind, wie bereits der alte Physiologe Burdach betonte, die Mutter z. B. durch das
               Wimmern ihres Kindes, der Müller durch das Stehenbleiben seiner Mühle, die
               meisten Menschen durch den leisen Anruf bei ihrem Namen. Diese Wache
               haltende Aufmerksamkeit wendet sich nun auch den inneren Wunschreizen aus dem Verdrängten zu und bildet mit ihnen den Traum, der als
               Kompromiß gleichzeitig beide Instanzen befriedigt. Der Traum schafft eine Art
               von psychischer Erledigung für den unterdrückten oder mit Hilfe des Verdrängten
               geformten Wunsch, indem er ihn als erfüllt hinstellt; er genügt aber auch
               der anderen Instanz, indem er die Fortsetzung des Schlafes gestattet. Unser
               Ich benimmt sich dabei gerne wie ein Kind, es schenkt den Traumbildern Glauben,
               als ob es sagen wollte: Ja, ja, du hast recht, aber laß mich schlafen. Die
               Geringschätzung, die wir, erwacht, dem Traume entgegenbringen, und die sich auf
               die Verworrenheit und scheinbare Unlogik des Traumes beruft, ist
                  wahrscheinlich nichts anderes als das Urteil unseres schlafenden
               Ichs über die Regungen aus dem Verdrängten, das sich mit besserem Rechte
               auf die motorische Ohnmacht dieser Schlafstörer stützt. Dies geringschätzige
               Urteil wird uns mitunter selbst im Schlafe bewußt; wenn der Trauminhalt
               allzusehr über die Zensur hinaus-
        \pend
    
         
            
            
            
        \pstart
        geht, denken wir: Es ist ja nur ein Traum — und schlafen weiter.
        \pend
    
            
        \pstart
        Es ist kein Einwand gegen diese Auffassung, wenn es auch für den Traum
               Grenzfälle gibt, in denen er seine Funktion, den Schlaf vor Unterbrechung zu
               bewahren, nicht mehr festhalten kann — wie beim Angsttraum — und sie gegen die
               andere ihn rechtzeitig aufzuheben, vertauscht. Er verfährt dabei auch nur
               wie der gewissenhafte Nachtwächter, der zunächst seine Pflicht tut, indem er
               Störungen zur Ruhe bringt, um die Bürgerschaft nicht zu wecken, dann aber seine
               Pflicht damit fortsetzt, die Bürgerschaft selbst zu wecken, wenn ihm
               die Ursachen der Störung bedenklich scheinen und er mit ihnen allein nicht
               fertig wird.
        \pend
    
            
        \pstart
        Besonders deutlich wird eine solche Funktion des Traumes, wenn an den
               Schlafenden ein Anreiz zu Sinnesempfindungen herantritt. Daß Sinnesreize,
               während des Schlafzustandes angebracht, den Inhalt der Träume beeinflussen, ist
               allgemein bekannt, läßt sich experimentell nachweisen und gehört zu den
               wenigen sicheren, aber arg überschätzten, Ergebnissen der ärztlichen
               Forschung über den Traum. Es hat sich aber an diese Ermittlung ein
               bisher unlösbares Rätsel geknüpft. Der Sinnesreiz, den der Experimentator auf
               den Schlafenden einwirken läßt, erscheint im Traume nämlich nicht richtig
               erkannt, sondern unterliegt irgend einer von unbestimmt vielen Deutungen, deren
                  Determinierung der psychischen Willkür überlassen schien.
               Psychische Willkür gibt es natürlich nicht. Der Schlafende kann gegen einen
               Sinnenreiz von außen auf mehrfache Weise reagieren. Entweder er erwacht oder es
               gelingt ihm, den Schlaf trotzdem fortzusetzen. Im letzteren Falle kann er sich
               des Traumes bedienen, um den äußeren Reiz fortzuschaffen, und zwar wiederum
               auf mehr als eine Weise. Er kann z. B. den Reiz aufheben, indem er eine
               Situation träumt, die mit ihm ganz und gar unverträglich ist. So benahm sich z.
               B. ein Schläfer,
        \pend
    
         
            
            
            
        \pstart
        den ein schmerzhafter Abszeß am Perineum stören wollte. Er träumte,
               daß er auf einem Pferd reite, benutzte dabei den Breiumschlag, der
               ihm den Schmerz lindern sollte, als Sattel und kam so über die Störung hinweg.
               Oder aber, was der häufigere Fall ist, der äußere Reiz erfährt eine Umdeutung,
               die ihn in den Zusammenhang eines eben auf seine Erfüllung lauernden
               verdrängten Wunsches einfügt, ihn so seiner Realität beraubt und wie ein Stück
               des psychischen Materials behandelt. So träumt jemand, er habe ein Lustspiel
               geschrieben, das eine bestimmte Grundidee verkörpert, es werde im Theater
               aufgeführt, der erste Akt sei vorüber und finde rasenden Beifall. Es wird
               fürchterlich geklatscht . . . Es muß hier dem Träumer gelungen sein, seinen
               Schlaf über die Störung hinaus zu verlängern, denn als er erwachte, hörte er das
               Geräusch nicht mehr, urteilte aber mit gutem Recht, es müßte ein Teppich oder
               Betten geklopft worden sein. — Die Träume, die sich unmittelbar vor dem
               Wecken durch ein lautes Geräusch einstellen, haben alle noch den Versuch
               gemacht, den erwarteten Weckreiz durch eine andere Erklärung abzuleugnen und den
               Schlaf noch um ein Weilchen zu verlängern.
        \pend
    
         
            
            
            \pstart\eledsection*{XII}\pend
            
        \pstart
        Wer an dem Gesichtspunkte der Zensur als dem Hauptmotiv der Traumentstellung festhält, der wird nicht befremdet
               sein, aus den Ergebnissen der Traumdeutung zu erfahren, daß die meisten Träume
               der Erwachsenen durch die Analyse auf
               erotische Wünsche zurückgeführt werden. Diese
               Behauptung zielt nicht auf die Träume von unverhüllt sexuellem Inhalt, die
               wohl allen Träumern aus eigenem Erleben bekannt sind und gewöhnlich allein als
               „sexuelle Träume“ beschrieben werden. Solche Träume bieten noch immer des
               Befremdenden genug durch die Auswahl der Personen, die sie zu
               Sexualobjekten machen, durch die Wegräumung aller Schranken, an denen der
               Träumer im wachen Leben seine geschlechtlichen Bedürfnisse haltmachen läßt,
               durch viele sonderbare an das sogenannt
               Perverse mahnende Einzelheiten. Die Analyse zeigt
               aber, daß sehr viele andere Träume, die in ihrem manifesten Inhalt nichts
               Erotisches erkennen lassen, durch die Deutungsarbeit als sexuelle
               Wunscherfüllungen entlarvt werden, und daß anderseits sehr viele von der
               Denkarbeit des Wachens als „Tagesreste“ erübrigte Gedanken zu ihrer
               Darstellung im Traum nur durch die Zuhilfenahme verdrängter
               erotischer Wünsche gelangen.
        \pend
    
            
        \pstart
        Zur Aufklärung dieses theoretisch nicht postulierten Sachverhaltes sei darauf
               hingewiesen, daß keine andere Gruppe von Trieben eine so weitgehende
               Unterdrückung durch die Anforderung der Erziehung zur Kultur erfahren hat wie
               gerade die sexuellen, daß
        \pend
    
         
            
            
            
        \pstart
        aber auch die sexuellen Triebe sich bei den meisten Menschen der
               Beherrschung durch die höchsten Seeleninstanzen am ehesten zu entziehen
               verstehen. Seitdem wir die in ihren Äußerungen oft so unscheinbare, regelmäßig
               übersehene und mißverstandene
               infantile Sexualität kennen gelernt haben, sind
               wir berechtigt zu sagen, daß fast jeder Kulturmensch die infantile
               Gestaltung des Sexuallebens in irgend einem Punkte festgehalten hat, und
               begreifen so, daß die verdrängten infantilen Sexualwünsche die
               häufigsten und stärksten Triebkräfte für die Bildung der Träume ergeben.1
        \pend
    
            
        \pstart
        Wenn es dem Traume, welcher erotische Wünsche zum Ausdrucke bringt,
               gelingen kann, in seinem manifesten Inhalt harmlos asexuell zu
               erscheinen, so kann dies nur auf eine Weise möglich werden. Das Material von
               sexuellen Vorstellungen darf nicht als solches dargestellt werden, sondern muß
               im Trauminhalt durch Andeutungen, Anspielungen und ähnliche
               Arten der indirekten Darstellung ersetzt werden, aber zum Unterschied von
               anderen Fällen indirekter Darstellung muß die im Traum verwendete der
               unmittelbaren Verständlichkeit entzogen sein. Man hat sich gewöhnt, die
               Darstellungsmittel, welche diesen Bedingungen entsprechen, als Symbole des durch sie Dargestellten zu
               bezeichnen. Ein besonderes Interesse hat sich ihnen zugewendet, seitdem man
               bemerkt hat, daß die Träumer derselben Sprache sich der nämlichen
               Symbole bedienen, ja, daß in einzelnen Fällen die Symbolgemeinschaft über die
                  Sprachgemeinschaft hinausreicht. Da die Träumer die Bedeutung
               der von ihnen verwendeten Symbole selbst nicht kennen, bleibt es zunächst
               rätselhaft, woher deren Beziehung zu dem durch sie Ersetzten und Bezeichneten
               rührt. Die Tatsache selbst ist aber unzweifelhaft und wird für die Technik der
               Traumdeutung bedeutsam, denn mit Hilfe einer Kenntnis der
               Traumsymbolik ist es möglich,
        \pend
    
            
        \footnote{1) Vergl. hiezu des
               Verfassers „Drei Abhandlungen zur Sexualtheorie“ 1905, fünfte Auflage 1922.
               (Ges. Schriften, Bd. V.)}
    
         
            
            
            
        \pstart
        den Sinn einzelner Elemente des Trauminhaltes, oder einzelner Stücke
               des Traumes, oder mitunter selbst ganzer Träume, zu verstehen, ohne den Träumer
               nach seinen Einfällen befragen zu müssen. Wir nähern uns so dem populären Ideal
               einer Traumübersetzung und greifen anderseits auf die Deutungstechnik
               der alten Völker zurück, denen Traumdeutung mit Deutung durch Symbolik
               identisch war.
        \pend
    
            
        \pstart
        Wiewohl die Studien über die Traumsymbole von einem Abschluß noch weit entfernt
               sind, können wir doch eine Reihe von allgemeinen Behauptungen und von speziellen
               Angaben über dieselben mit Sicherheit vertreten. Es gibt Symbole, die fast
               allgemein eindeutig zu übersetzen sind, so bedeuten Kaiser und Kaiserin (König
               und Königin) die Eltern, Zimmer stellen Frauen(zimmer) dar, die Ein- und
               Ausgänge derselben die Körperöffnungen. Die größte Zahl der
               Traumsymbole dient zur Darstellung von Personen, Körperteilen und
               Verrichtungen, die mit erotischem Interesse betont sind, insbesondere können
               die Genitalien durch eine Anzahl von oft sehr überraschenden Symbolen
               dargestellt werden und finden sich die mannigfaltigsten Gegenstände zur
               symbolischen Bezeichnung der Genitalien verwendet. Wenn scharfe
               Waffen, lange und starre Objekte wie Baumstämme und Stöcke, das männliche
               Genitale; Schränke, Schachteln, Wagen, Öfen den Frauenleib im Traume
               vertreten, so ist uns das Tertium comparationis, das Gemeinsame dieser
               Ersetzungen, ohne weiteres verständlich, aber nicht bei allen Symbolen wird uns
               das Erfassen der verbindenden Beziehungen so leicht gemacht. Symbole wie das der
               Stiege und des Steigens für den Sexualverkehr, der Krawatte für das männliche
               Glied, des Holzes für den Frauenleib fordern unseren Unglauben heraus, so
               lange wir nicht die Einsicht in die Symbolbeziehung auf anderen Wegen gewonnen
               haben. Eine ganze Anzahl der Traumsymbole ist übrigens bisexuell, kann je nach
               dem Zusammenhange auf das männliche oder auf das weibliche Genitale bezogen
               werden.
        \pend
    
         
            
            
            
        \pstart
        Es gibt Symbole von universeller Verbreitung, die man bei allen Träumern eines
               Sprach- und Bildungskreises antrifft, und andere von höchst eingeschränktem,
               individuellem Vorkommen, die sich ein Einzelner aus seinem Vorstellungsmaterial
               gebildet hat. Unter den ersteren unterscheidet man solche, deren Anspruch
               auf Vertretung des Sexuellen durch den Sprachgebrauch ohne weiteres
               gerechtfertigt wird (wie z. B. die aus dem Ackerbau stammenden, vgl.
               Fortpflanzung, Samen), von anderen, deren Beziehung zum Sexuellen in die
               ältesten Zeiten und dunkelsten Tiefen unserer Begriffsbildung hinabzureichen
               scheint. Die symbolbildende Kraft ist für beide oben gesonderten Arten von
               Symbolen in der Gegenwart nicht erloschen. Man kann beobachten, daß neu
               erfundene Gegenstände (wie das Luftschiff) sofort zu universell gebräuchlichen
               Sexualsymbolen erhoben werden.
        \pend
    
            
        \pstart
        Es wäre übrigens irrtümlich zu erwarten, eine noch gründlichere
               Kenntnis der Traumsymbolik (der „Sprache des Traumes“) könnte uns von der
               Befragung des Träumers nach seinen Einfällen zum Traume unabhängig
               machen und uns gänzlich zur Technik der antiken Traumdeutung zurückführen.
               Abgesehen von den individuellen Symbolen und den Schwankungen im Gebrauch
               der universellen, weiß man nie, ob ein Element des Trauminhaltes symbolisch oder
               im eigentlichen Sinne zu deuten ist, und weiß man mit Sicherheit, daß nicht
               aller Inhalt des Traumes symbolisch zu deuten ist. Die Kenntnis der Traumsymbolik wird uns immer nur die Übersetzung einzelner
               Bestandteile des Trauminhaltes vermitteln und wird die Anwendung der
               früher gegebenen technischen Regeln nicht überflüssig machen. Sie wird aber als
               das wertvollste Hilfsmittel zur Deutung gerade dort eintreten, wo die Einfälle
               des Träumers versagen oder ungenügend werden.
        \pend
    
            
        \pstart
        Die Traumsymbolik erweist sich als unentbehrlich auch für das Verständnis der
               sogenannten „typischen“ Träume der
        \pend
    
         
            
            
            
        \pstart
        Menschen und der „wiederkehrenden“ Träume des Einzelnen. Wenn die
               Würdigung der symbolischen Ausdrucksweise des Traumes in dieser kurzen
               Darstellung so unvollständig aufgefallenausgefallen ist, so rechtfertigt sich diese
               Vernachlässigung durch den Hinweis auf eine Einsicht, die zu dem
               Wichtigsten gehört, was wir über diesen Gegenstand aussagen können. Die
               Traumsymbolik führt weit über den Traum hinaus; sie gehört nicht dem Traume
               zu eigen an, sondern beherrscht in gleicher Weise die Darstellung in den
               Märchen, Mythen und Sagen, in den Witzen und im Folklore. Sie gestattet uns, die
               innigen Beziehungen des Traumes zu diesen Produktionen zu verfolgen; wir müssen
               uns aber sagen, daß sie nicht von der Traumarbeit hergestellt wird, sondern
               eine Eigentümlichkeit — wahrscheinlich unseres unbewußten Denkens
               ist, welches der Traumarbeit das Material zur Verdichtung, Verschiebung und
               Dramatisierung liefert.1
        \pend
    
            
        \footnote{1) Weiteres über die
               Traumsymbolik findet man außer in den alten Schriften zur Traumdeutung (Artemidorus von Daldis, Scherner\edindex[person]{Person Nr. 03500}, Das Leben des
               Traumes 1861) in der „Traumdeutung“ des Verfassers, in den mythologischen
               Arbeiten der psychoanalytischen Schule und auch in den Arbeiten von W. Stekel („Die Sprache des Traumes“ 1911).}
    
         
            
            
            \pstart\eledsection*{XIII}\pend
            
        \pstart
        Ich erhebe weder den Anspruch, hier auf alle Traumprobleme Licht geworfen, noch
               die hier erörterten überzeugend erledigt zu haben. Wer sich für den ganzen
               Umfang der Traumliteratur interessiert, der sei auf das Buch von Sante de Sanctis:
               I sogni, Torino 1899, verwiesen; wer die eingehendere
                  Begründung der von mir vorgetragenen Auffassung des Traumes
               sucht, der wende sich an meine Schrift: Die
                  Traumdeutung, Leipzig und Wien 1900.1 Ich werde nur noch darauf hinweisen, in welcher Richtung die
               Fortsetzung meiner Darlegungen über die Traumarbeit zu verfolgen ist.
        \pend
    
            
        \pstart
        Wenn ich als die Aufgabe einer Traumdeutung die Ersetzung des Traumes durch die
               latenten Traumgedanken, also die Auflösung dessen, was die
               Traumarbeit gesponnen hat, hinstelle, so werfe ich einerseits eine Reihe von
               neuen psychologischen Problemen auf, die sich auf den Mechanismus
               dieser Traumarbeit selbst wie auf die Natur und die Bedingungen der
               sogenannten Verdrängung beziehen; anderseits behaupte ich die Existenz der
               Traumgedanken, als eines sehr reichhaltigen Materiales psychischer Bildungen von
               höchster Ordnung und mit allen Kennzeichen normaler intellektueller Leistung
               versehen, welches Material sich doch dem Bewußtsein entzieht, bis es ihm durch
               den Trauminhalt entstellte Kunde gegeben hat. Solche Gedanken bin
               ich
        \pend
    
            
        \footnote{1) 1922 in siebenter Auflage
               erschienen. (Ges. Schriften, Bd. II und III.)}
    
         
            
            
            
        \pstart
        genötigt, bei jedermann als vorhanden anzunehmen, da ja fast alle
               Menschen, auch die normalsten, des Träumens fähig sind. An das Unbewußte der
               Traumgedanken, an dessen Verhältnis zum Bewußtsein und zur Verdrängung knüpfen
               die weiteren, für die Psychologie bedeutsamen Fragen an, deren Erledigung
               wohl aufzuschieben ist, bis die Analyse die Entstehung anderer psychopathischer Bildungen, wie der hysterischen Symptome und der
               Zwangsideen, klargelegt hat.
        \pend
    
         
        \endnumbering
  
        \section*{Kritischer Apparat}
        \doendnotes{A}
        
        \section*{Stellenkommentar}
        \doendnotes{B}
        
        \printindex[person]
        \printindex[kw]
        


        \end{document}
    