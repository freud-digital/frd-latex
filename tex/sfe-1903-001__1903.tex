
        \documentclass[twoside=true,titlepage=false,open=any, parskip=never, fontsize=10pt, headings=small, chapterprefix=false, appendixprefix=false]{scrbook}
        \addtolength{\oddsidemargin}{\evensidemargin}
        \setlength{\oddsidemargin}{.5\oddsidemargin}
        \setlength{\evensidemargin}{\oddsidemargin}
        \usepackage[paperheight=230mm, paperwidth=138mm, textwidth=100mm, textheight=185mm]{geometry}
        
        \usepackage{scrlayer-scrpage}
        \usepackage{hyphenat}
        \usepackage{fontspec}
        \usepackage{moresize}
        %\usepackage{ipa}
        \usepackage[english, ngerman]{babel}
        \usepackage[babel]{microtype}
        \usepackage{soul}
        \usepackage{scrhack}
        \usepackage{xpatch}
        \usepackage[nonewpage]{indextools}
        \usepackage{lettrine}
        \usepackage[series={A,B,C}]{reledmac}
        
        \setmainfont[Path=./fonts/,
        Extension=.ttf,
        UprightFont=*-Roman,
        BoldFont=*-Bold,
        ItalicFont=*-Italic,
        BoldItalicFont=*-BoldItalic]{Brill}
        
        \KOMAoptions{toc=chapterentrydotfill, toc=flat}
        \addtokomafont{chapterentrypagenumber}{\mdseries}
        \setkomafont{chapterentry}{\normalfont\mdseries}
        \setkomafont{partentry}{\normalfont\mdseries}
        \RedeclareSectionCommand[tocbeforeskip=0pt]{chapter}         
        
        \setlength{\skip\footins}{4mm plus 2mm} % Abstand Fussnote Text
        \interfootnotelinepenalty=10000 % Kein Seitenwechsel in Fuss
        
        %% Sperrung (Package Soul)
        %% Hier ist die Sperrung definiert. Sperrung erreicht man mit \so{gesperrtes Wort}
        \sodef\so{}{.14em}{.4em plus.1em minus .1em}{.4em plus.1em minus .1em}
        
        
        
        % Fußnoten linksbündig
        \deffootnote{1.5em}{1em}{% 
        \makebox[1.5em][l]{\thefootnotemark}%
        }
        
        
        % Fussnotenlineal (wobei für reledmac wohl was anderes gilt)
        \let\normalfootnoterule\footnoterule
        \setfootnoterule{0pt}
        \let\normalfootnoterule\footnoterule
        
        
        \setlength{\skip\footins}{8mm plus 2mm} % Abstand Fussnote Text
        \interfootnotelinepenalty=10000 % Kein Seitenwechsel in Fuss
        
        %% Kapitelüberschriften
        \renewcommand*{\raggedchapter}{\centering} 
        \renewcommand*{\raggedsection}{%
        \CenteringLeftskip=1cm plus 1em\relax 
        \CenteringRightskip=1cm plus 1em\relax 
        \Centering\footnotesize\thesection{}.\ }
        \setkomafont{section}{\footnotesize}
        \setkomafont{chapter}{\normalfont\Large}
        \renewcommand{\chapterpagestyle}{empty}%The first page in each chapter won't have any heading or footer, especially no page number
        
        % section ohne führende Kapitelnummer
        \renewcommand*\thesection{\arabic{section}}
        
        % Bildunterschrift ohne Nummer
        \renewcommand*{\figureformat}{}
        \renewcommand*{\captionformat}{}
        
        %% Zeilennummern
        \firstlinenum{0} \linenumincrement{5}
        \lineation{section} %Jeder Abschnitt wird durchnummeriert
        \renewcommand{\numlabfont}{\ssmall} %Schriftgröße Zeilennummern
        
        
        \newcommand{\Theight}{\dimexpr\fontcharht\font`W}
        \newcommand{\pbposition}{\depth}
        \newcommand{\pb}{\nobreak\hspace{0pt}\raisebox{-0.1em}{\raisebox{\pbposition}{\textnormal{HANSI4EVER}}}\nobreak\hspace{0pt}}
        
        \renewcaptionname{ngerman}{\contentsname}{Inhalt}           %Table of contents
        
        % FUSSNOTE
        %% Im Apparat f. und ff.
        \Xtwolines{f.}
        \Xtwolinesbutnotmore
        
        %% Zeilennummerierung Abstand zum Lemma
        \Xboxlinenum{5mm}
        
        %% Bei zwei Apparateinträgen in einer Zeile wird nur beim ersten Mal die Zeile gezählt
        \Xnumberonlyfirstinline
        \Xnumberonlyfirstintwolines
        \Xinplaceofnumber{1em}
        \Xhangindent{1em}
        
        % ENDNOTEN
        \Xendbeforepagenumber{} 
        \Xendparagraph[A]
        \Xendsep{}
        \Xendafterpagenumber[A]{, }
        \Xendnotenumfont[A]{\tiny}
        \Xendlineprefixsingle[A]{\tiny}
        \Xendlineprefixmore[A]{\tiny}
        \Xendhangindent[A]{4em}
        
        \Xendboxlinenum[A]{0em}
        \Xendlemmaseparator{$\rbracket$}
        \Xendnotefontsize[A]{\footnotesize}
        \Xendhangindent[A]{0em}
        \Xendlemmafont[A]{\normalfont}
        \Xendlemmafont[B]{\bfseries}
        \Xendnotefontsize[B]{\footnotesize}
        \Xendnotenumfont{\footnotesize}
        \Xendlineprefixsingle[C]{\tiny}
        \Xendlineprefixmore[C]{\tiny}
        \Xendlemmadisablefontselection
        \Xendlemmafont{\itshape}
        \Xendlinerangeseparator{\tiny{--}}
        \Xendhangindent{4em}
        \Xendboxlinenum{3.6em}
        \Xendafternumber{0.4em}
        \Xendboxlinenumalign{R}
        
        \makeindex[name=kw,title=Schlagworteregister,columns=2]
        \makeindex[name=person,title=Personenregister,columns=2]
        
        \begin{document}

        \beginnumbering
         
        \pstart
        Neue Freie Presse 8. Februar 1903 
        \pend
     
        \pstart
        [Dr. Georg Biedenkapp. „Im Kampfe gegen Hirnbacillen“. Berlin, 1902.] Hinter diesem wenig ansprechenden Titel birgt sich das Buch eines tapferen Mannes, der dem Leser viel Beherzigenswertes zu sagen weiß. Mehr von dem Inhalt verräth der Untertitel des Werkes: „Eine Philosophie der kleinen Worte“. Der Autor kämpft nämlich gegen jene „zu Vieles aus- oder einschließenden Wörtchen und Wortformen“, welche bei denen, die sie mit Vorliebe zu gebrauchen pflegen, eine schädliche Neigung zu „exclusiven oder superlativen Urtheilen“ bekunden. Selbstverständlich — auch dieses Wort würde unser Autor beanstanden — gilt der Kampf nicht jenen harmlosen Worten, sondern der Neigung, sich an ihnen zu berauschen und der so gewonnenen Hebung der Darstellung zuliebe an die nothwendigen Einschränkungen seiner Aussage, wie an die unvermeidliche Bedingtheit der eigenen Urtheile zu vergessen. Es dient wirklich zur nützlichen Mahnung, wenn Einem vorgehalten wird, wie Vieles als „selbstverständlich“ oder als „unsinnig“ von den Menschen einer früheren Generation bezeichnet wurde, was uns heute umgekehrt als unsinnig oder als selbstverständlich gilt. Oder wenn wir an einer Reihe gut gewählter Beispiele ersehen, welche Einengung ihres Gesichtskreises selbst bedeutende Schriftsteller sich in Folge ihres Mißbrauches von Superlativen vorwerfen lassen müssen. Die Mahnung zur Nüchternheit in Urtheil und Ausdruck dient unserem Autor indeß nur als Ausgangspunkt zu weiteren Erörterungen über andere „Denkfehler“ der Menschen, über den Mittelpunktswahn, den Glauben, über die atheistische Moral und dergleichen. In all diesen Bemerkungen zeigt sich das ehrliche Bestreben des Autors, Ernst zu machen mit der Durchführung jener Weltanschauung, die uns durch die Ergebnisse der modernen Wissenschaft, im Besonderen der Entwicklungslehre, aufgenöthigt wird. Es ist sehr viel psychologisch Richtiges dabei, und manche Wahrheit von der Art, die schon oft gesagt worden ist, aber nicht oft genug wiederholt werden kann. Der Autor hat sich die undankbare Aufgabe gestellt, „die Menschen zu bessern und zu bekehren“ auf dem Wege nüchterner Beeinflussung, ohne sie durch Humor zum Lachen bewegen oder durch Leidenschaft mit fortreißen zu wollen. Wünschen wir ihm dazu den besten Erfolg! Professor Sigmund Freud.
        \pend
     
        \endnumbering
  
        \section*{Kritischer Apparat}
        \doendnotes{A}
        
        \section*{Stellenkommentar}
        \doendnotes{B}
        
        \printindex[person]
        \printindex[kw]
        


        \end{document}
    