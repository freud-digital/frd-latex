
        \documentclass[twoside=true,titlepage=false,open=any, parskip=never, fontsize=10pt, headings=small, chapterprefix=false, appendixprefix=false]{scrbook}
        \addtolength{\oddsidemargin}{\evensidemargin}
        \setlength{\oddsidemargin}{.5\oddsidemargin}
        \setlength{\evensidemargin}{\oddsidemargin}
        \usepackage[paperheight=230mm, paperwidth=138mm, textwidth=100mm, textheight=185mm]{geometry}
        
        \usepackage{scrlayer-scrpage}
        \usepackage{hyphenat}
        \usepackage{fontspec}
        \usepackage{moresize}
        %\usepackage{ipa}
        \usepackage[english, ngerman]{babel}
        \usepackage[babel]{microtype}
        \usepackage{soul}
        \usepackage{scrhack}
        \usepackage{xpatch}
        \usepackage[nonewpage]{indextools}
        \usepackage{lettrine}
        \usepackage[series={A,B,C}]{reledmac}
        
        \setmainfont[Path=./fonts/,
        Extension=.ttf,
        UprightFont=*-Roman,
        BoldFont=*-Bold,
        ItalicFont=*-Italic,
        BoldItalicFont=*-BoldItalic]{Brill}
        
        \KOMAoptions{toc=chapterentrydotfill, toc=flat}
        \addtokomafont{chapterentrypagenumber}{\mdseries}
        \setkomafont{chapterentry}{\normalfont\mdseries}
        \setkomafont{partentry}{\normalfont\mdseries}
        \RedeclareSectionCommand[tocbeforeskip=0pt]{chapter}         
        
        \setlength{\skip\footins}{4mm plus 2mm} % Abstand Fussnote Text
        \interfootnotelinepenalty=10000 % Kein Seitenwechsel in Fuss
        
        %% Sperrung (Package Soul)
        %% Hier ist die Sperrung definiert. Sperrung erreicht man mit \so{gesperrtes Wort}
        \sodef\so{}{.14em}{.4em plus.1em minus .1em}{.4em plus.1em minus .1em}
        
        
        
        % Fußnoten linksbündig
        \deffootnote{1.5em}{1em}{% 
        \makebox[1.5em][l]{\thefootnotemark}%
        }
        
        
        % Fussnotenlineal (wobei für reledmac wohl was anderes gilt)
        \let\normalfootnoterule\footnoterule
        \setfootnoterule{0pt}
        \let\normalfootnoterule\footnoterule
        
        
        \setlength{\skip\footins}{8mm plus 2mm} % Abstand Fussnote Text
        \interfootnotelinepenalty=10000 % Kein Seitenwechsel in Fuss
        
        %% Kapitelüberschriften
        \renewcommand*{\raggedchapter}{\centering} 
        \renewcommand*{\raggedsection}{%
        \CenteringLeftskip=1cm plus 1em\relax 
        \CenteringRightskip=1cm plus 1em\relax 
        \Centering\footnotesize\thesection{}.\ }
        \setkomafont{section}{\footnotesize}
        \setkomafont{chapter}{\normalfont\Large}
        \renewcommand{\chapterpagestyle}{empty}%The first page in each chapter won't have any heading or footer, especially no page number
        
        % section ohne führende Kapitelnummer
        \renewcommand*\thesection{\arabic{section}}
        
        % Bildunterschrift ohne Nummer
        \renewcommand*{\figureformat}{}
        \renewcommand*{\captionformat}{}
        
        %% Zeilennummern
        \firstlinenum{0} \linenumincrement{5}
        \lineation{section} %Jeder Abschnitt wird durchnummeriert
        \renewcommand{\numlabfont}{\ssmall} %Schriftgröße Zeilennummern
        
        
        \newcommand{\Theight}{\dimexpr\fontcharht\font`W}
        \newcommand{\pbposition}{\depth}
        \newcommand{\pb}{\nobreak\hspace{0pt}\raisebox{-0.1em}{\raisebox{\pbposition}{\textnormal{HANSI4EVER}}}\nobreak\hspace{0pt}}
        
        \renewcaptionname{ngerman}{\contentsname}{Inhalt}           %Table of contents
        
        % FUSSNOTE
        %% Im Apparat f. und ff.
        \Xtwolines{f.}
        \Xtwolinesbutnotmore
        
        %% Zeilennummerierung Abstand zum Lemma
        \Xboxlinenum{5mm}
        
        %% Bei zwei Apparateinträgen in einer Zeile wird nur beim ersten Mal die Zeile gezählt
        \Xnumberonlyfirstinline
        \Xnumberonlyfirstintwolines
        \Xinplaceofnumber{1em}
        \Xhangindent{1em}
        
        % ENDNOTEN
        \Xendbeforepagenumber{} 
        \Xendparagraph[A]
        \Xendsep{}
        \Xendafterpagenumber[A]{, }
        \Xendnotenumfont[A]{\tiny}
        \Xendlineprefixsingle[A]{\tiny}
        \Xendlineprefixmore[A]{\tiny}
        \Xendhangindent[A]{4em}
        
        \Xendboxlinenum[A]{0em}
        \Xendlemmaseparator{$\rbracket$}
        \Xendnotefontsize[A]{\footnotesize}
        \Xendhangindent[A]{0em}
        \Xendlemmafont[A]{\normalfont}
        \Xendlemmafont[B]{\bfseries}
        \Xendnotefontsize[B]{\footnotesize}
        \Xendnotenumfont{\footnotesize}
        \Xendlineprefixsingle[C]{\tiny}
        \Xendlineprefixmore[C]{\tiny}
        \Xendlemmadisablefontselection
        \Xendlemmafont{\itshape}
        \Xendlinerangeseparator{\tiny{--}}
        \Xendhangindent{4em}
        \Xendboxlinenum{3.6em}
        \Xendafternumber{0.4em}
        \Xendboxlinenumalign{R}
        
        \makeindex[name=kw,title=Schlagworteregister,columns=2]
        \makeindex[name=person,title=Personenregister,columns=2]
        
        \begin{document}

        \beginnumbering
        
            \pstart[\section{Vorwort}]
            Textentstehung und Publikationsgeschichte
            \pend
            \pstart
            
                  Als „Mitteilung des Autors“, wie in Klammern im
                  Untertitel angegeben wird, wurde der kurze Text Über die Freud’sche
                     Psychoanalytische Methode zuerst in  Leopold Löwenfelds Band über Psychische Zwangserscheinungen veröffentlicht,
                  welcher 1904 von Freud rezensiert wurde (vgl. hier 1904-006).
                  Zuvor war Freuds Schrift Über den Traum (1901) 
                  in der Reihe Grenzfragen des
                  Nerven- und Seelenlebens publiziert worden, welche von Leopold
                  Löwenfeld und Hans Kurella
                  herausgegeben wurde.Ein Blick auf das \textbf{Inhaltsverzeichnis} der Reihe zeigt die thematischen und
                  personellen Verschränkungen auf dem Gebiet der frühen Traumforschung, Sexual- und
                  Psychopathologie um 1900.
                  Auf eine fachliche Kontroverse im Jahr 1895 zwischen
                  Freud und Löwenfeld die Angstneurose betreffend, verweist Tögel, diese konnte aber
                  beigelegt werden (Vgl. Freud: Zur Kritik der
                  „Angstneurose“, in Wiener klinische Rundschau, 9. Jg. (1885), Nr.
                  27, S. 417-419, Nr. 28, S. 435-437, Nr. 29, S. 451-452)(Tögel, Bd. 9, S. 111). 
                  Der Erstveröffentlichung des Textes Über die Freud’sche Psychoanalytischen
                  Methode war nach Ende des Beitrags ein Kommentar von Löwenfeld
                  beigegeben.              
                  Löwenfeld berücksichtigte die Psychoanalyse als Verfahren im Kontext der
                  Zwangserkrankungen, denen er sich eingehend widmete, war allerdings der Meinung,
                  dass Freud seine Methode seit den Studien über Hysterie aus dem Jahr 1895 so
                  grundlegend verändert habe, dass diese keine
                  Vorstellung von dem Wesen der Methode mehr geben
                  Löwenfeld, Zwangserscheinungen, S. 545 können. So
                  entschied er sich, Freud selbst zu Wort kommen zu lassen und leitet mit einem Dank
                  zu Freuds Ausführungen über: Mit Rücksicht auf
                  diese Sachlage bin ich, da eine anderweitige Publikation über den Gegenstand
                  nicht vorliegt, dem Autor sehr verpflichtet, dass er mir auf Ersuchen
                  nachstehendes Exposé über die gegenwärtige Gestaltung seines Verfahrens zur
                  Veröffentlichung an dieser Stelle überliess.
                  Löwenfeld, Zwangserscheinungen, ebd.
                  Ab 1911? wurde der Text Über die Freud’sche Psychoanalytische Methode in die
                     Sammlung kleiner Schriften zur Neurosenlehre aufgenommen.
               
            \pend
        
        \beforeeledchapter
    
            
            
            
        \pstart[\section{Die Freudsche psychoanalytische Methode}]
        
               
                  \edtext{DIE FREUD’SCHE}{\lemma{\textbf{DIE FREUD’SCHE}}\Aendnote{Freudsche S2S3S4S5C }} PSYCHOANALYTISCHE METHODE
               
            
        \pend
        \beforeeledchapter
    
            
        \pstart
        „Die eigentümliche Methode der Psychotherapie, die \so{Freud} ausübt und als \edtext{Psychoanalyse\edindex[kw]{Schlagwort Nr. 2113}}{\lemma{\textbf{Psychoanalyse\edindex[kw]{Schlagwort Nr. 2113}}}\Aendnote{Psycho-Analyse E }} bezeichnet, ist aus dem sogenanntenkathartischen Verfahren\edindex[kw]{Schlagwort Nr. 3415} hervorgegangen,
               über welches er seinerzeit in den „Studien über Hysterie“ 1895 in
               Gemeinschaft mit
               J. Breuer\edindex[person]{Person Nr. 40}\edtext{}{\Bendnote{\textbf{Breuer} Josef Breuer
                  (1842-1925), österr. Internist und Physiologe, veröffentl. gem. mit Freud u.a.
                     Studien über Hysterie .}} berichtet
               hat. Diekathartische
                  Therapie\edindex[kw]{Schlagwort Nr. 3415} war eine Erfindung \so{Breuers}, der mit ihrer Hilfe zuerst etwa ein Dezennium vorher
               eine hysterische Kranke hergestellt und dabei Einsicht in die Pathogenese\edindex[kw]{Schlagwort Nr. 1983}\edtext{}{\Bendnote{\textbf{Pathogenese} Krankheitsentwicklung}} ihrer Symptome gewonnen
               hatte. Infolge einer persönlichen Anregung \so{Breuers} nahm dann \so{Freud} das Verfahren
               wieder auf und erprobte es an einer \edtext{größeren}{\lemma{\textbf{größeren}}\Aendnote{grösseren E }} Anzahl von Kranken.
        \pend
    
            
        \pstart
        Das kathartische Verfahren setzte voraus, \edtext{daß}{\lemma{\textbf{daß}}\Aendnote{dass E }} der Patient hypnotisierbar\edtext{sei,}{\lemma{\textbf{sei,}}\Aendnote{sei und S2ES1S3S4 }} und beruhte auf der Erweiterung des \edtext{Bewußtseins}{\lemma{\textbf{Bewußtseins}}\Aendnote{Bewusstseins E }}, die in der Hypnose\edindex[kw]{Schlagwort Nr. 1219} eintritt. \edtext{Es}{\lemma{\textbf{Es}}\Aendnote{Er E }} setzte sich die Beseitigung der Krankheitssymptome zum Ziele und
               erreichte dies, indem \edtext{es}{\lemma{\textbf{es}}\Aendnote{er E }} den Patienten sich in den psychischen Zustand zurückversetzen \edtext{ließ}{\lemma{\textbf{ließ}}\Aendnote{liess E }}, in welchem das Symptom\edindex[kw]{Schlagwort Nr. 2643} zum \edtext{erstenmal}{\lemma{\textbf{erstenmal}}\Aendnote{erstenmale E ersten Male S1S2S3S4 }} aufgetreten war. Es tauchten dann bei dem hypnotisierten Kranken
               Erinnerungen, Gedanken und Impulse auf, die in seinem \edtext{Bewußtsein}{\lemma{\textbf{Bewußtsein}}\Aendnote{Bewusstsein E }} bisher ausgefallen waren, und wenn er diese seine seelischen Vorgängeunter intensiven \edtext{Affektäußerungen}{\lemma{\textbf{Affektäußerungen}}\Aendnote{Affektäusserungen E }} dem Arzte mitgeteilt hatte, war das Symptom überwunden, die Wiederkehr
               desselben aufgehoben. Diese \edtext{regelmäßig}{\lemma{\textbf{regelmäßig}}\Aendnote{regelmässig E }} zu wiederholende Erfahrung erläuterten die beiden Autoren in ihrer
               gemeinsamen Arbeit dahin, \edtext{daß}{\lemma{\textbf{daß}}\Aendnote{dass E }} das Symptom an Stelle von unterdrückten und nicht zum \edtext{Bewußtsein}{\lemma{\textbf{Bewußtsein}}\Aendnote{Bewusstsein E }} gelangten psychischen Vorgängen stehe, also eine Umwandlung
               („Konversion“) der letzteren darstelle. Die therapeutische Wirksamkeit ihres Verfahrens erklärten sie sich aus der Abfuhr des bis dahin gleichsam „eingeklemmten“ Affektes, der an den
                  unterdrückten seelischen Aktionen gehaftet hatte („Abreagieren“).
               Das einfache Schema des therapeutischen\edtext{Eingriffes}{\lemma{\textbf{Eingriffes}}\Aendnote{Eingriffs ES1 }} komplizierte sich aber nahezu \edtext{allemal}{\lemma{\textbf{allemal}}\Aendnote{alle Male ES1S2S3S4 }}, indem sich zeigte, \edtext{daß}{\lemma{\textbf{daß}}\Aendnote{dass E }} nicht ein einzelner („traumatischer“) Eindruck, sondern meist eine schwer
               zu übersehende Reihe von solchen an der Entstehung des Symptoms
               beteiligt sei.
        \pend
    


            
        \pstart
        Der Hauptcharakter der kathartischen Methode, der sie \edtext{in}{\lemma{\textbf{in}}\Aendnote{im S3S4 }} Gegensatz zu allen anderen Verfahren der Psychotherapie setzt, liegt
               also darin, \edtext{daß}{\lemma{\textbf{daß}}\Aendnote{dass E }} bei ihr die therapeutische Wirksamkeit nicht einem suggestiven Verbot des
               Arztes übertragen wird. Sie erwartet vielmehr, \edtext{daß}{\lemma{\textbf{daß}}\Aendnote{dass E }} die Symptome von selbst verschwinden werden, wenn es dem Eingriff, der
               sich auf gewisse Voraussetzungen über den psychischen \edtext{Mechanismus}{\lemma{\textbf{Mechanismus}}\Aendnote{Mechanismns S3 }} beruft, gelungen ist, seelische Vorgänge zu einem \edtext{andern}{\lemma{\textbf{andern}}\Aendnote{anderen ES1 }} als dem bisherigen \edtext{Verlaufe}{\lemma{\textbf{Verlaufe}}\Aendnote{Verlauf ES1 }} zu bringen, der in die Symptombildung eingemündet hat.
        \pend
    
            
        \pstart
        Die Abänderungen, welche \so{Freud} an dem kathartischen
                  Verfahren \so{Breuers} vornahm, waren zunächst Änderungen der Technik; diese brachten
               aber neue Ergebnisse und haben in weiterer Folge zu einer andersartigen, wiewohl
               der früheren nicht \edtext{widersprechenden}{\lemma{\textbf{widersprechenden}}\Aendnote{widersprechenden, S1S2S3S4 }} Auffassung der therapeutischen Arbeit genötigt.
        \pend
    
            
        \pstart
        Hatte die kathartische Methode bereits auf die Suggestion\edindex[kw]{Schlagwort Nr. 2631} verzichtet, so unternahm \so{Freud} den weiteren Schritt, auch die Hypnose aufzugeben. Er behandelt
                  gegenwärtig\edtext{}{\Bendnote{\textbf{gegenwärtig} Zur Zeit der
                  Entstehung des Textes: 1904}} seine Kranken,indem er sie ohne andersartige Beeinflussung eine bequeme Rückenlage auf
               einem Ruhebett einnehmen \edtext{läßt}{\lemma{\textbf{läßt}}\Aendnote{lässt E }}, während er
               \edtext{selbst,}{\lemma{\textbf{selbst,}}\Aendnote{selbst S2ES1S3S4 }} ihrem Anblick \edtext{entzogen,}{\lemma{\textbf{entzogen,}}\Aendnote{entzogen S2ES1S3S4 }} auf einem Stuhle hinter ihnen sitzt. Auch den \edtext{Verschluß}{\lemma{\textbf{Verschluß}}\Aendnote{Verschluss E }} der Augen fordert er von ihnen nicht und vermeidet jede Berührung sowie
               jede andere Prozedur, die an Hypnose mahnen könnte. Eine solche Sitzung verläuft
               also wie ein Gespräch zwischen zwei gleich wachen Personen, von denen die
               eine sich jede Muskelanstrengung und jeden ablenkenden Sinneseindruck
               erspart, die sie in der Konzentration ihrer Aufmerksamkeit auf ihre eigene
               seelische Tätigkeit stören könnten.
        \pend
    


            
        \pstart
        Da das \edtext{Hypnotisiertwerden}{\lemma{\textbf{Hypnotisiertwerden}}\Aendnote{Hypnotisirtwerden E }}, trotz aller Geschicklichkeit des
               \edtext{Arztes,}{\lemma{\textbf{Arztes,}}\Aendnote{Arztes E }} bekanntlich in der Willkür des Patienten \edtext{liegt,}{\lemma{\textbf{liegt,}}\Aendnote{liegt E }} und eine
               \edtext{große}{\lemma{\textbf{große}}\Aendnote{grosse E }} Anzahl neurotischer Personen durch kein Verfahren in Hypnose zu versetzen
               ist, so war durch den Verzicht auf die Hypnose die Anwendbarkeit des Verfahrens
               auf eine uneingeschränkte Anzahl von Kranken gesichert. \edtext{Anderseits}{\lemma{\textbf{Anderseits}}\Aendnote{Andererseits ES1 }} fiel die Erweiterung des \edtext{Bewußtseins}{\lemma{\textbf{Bewußtseins}}\Aendnote{Bewusstseins E }} weg, welche dem Arzt gerade jenes psychische Material an Erinnerungen und
               Vorstellungen geliefert hatte, mit dessen Hilfe sich die Umsetzung der Symptome
               und die Befreiung der Affekte vollziehen \edtext{ließ}{\lemma{\textbf{ließ}}\Aendnote{liess E }}. Wenn für diesen Ausfall kein Ersatz zu schaffen war, konnte
               auch von einer therapeutischen Einwirkung keine Rede sein.
        \pend
    
            
        \pstart
        Einen solchen völlig ausreichenden Ersatz fand nun \so{Freud} in den Einfällen der Kranken, \edtext{das heißt}{\lemma{\textbf{das heißt}}\Aendnote{d. h. ES1S2S3S4 }} in den ungewollten, meist als störend empfundenen und darum unter
               gewöhnlichen Verhältnissen beseitigten Gedanken, die den Zusammenhang einer
               beabsichtigten Darstellung zu durchkreuzen pflegen. Um sich dieser Einfälle zu
               bemächtigen, fordert er die Kranken auf, sich in ihren Mitteilungen gehen zu\edtext{lassen,}{\lemma{\textbf{lassen,}}\Aendnote{lassen E }} „wie man es etwa in einem \edtext{Gespräche}{\lemma{\textbf{Gespräche}}\Aendnote{Gespräch ES1 }}
               tut, bei welchem man aus dem
               Hundertsten in das Tausendste \edtext{gerät.“}{\lemma{\textbf{gerät.“}}\Aendnote{gerät“. ES1S2S3S4 }} Er schärft ihnen, ehe er sie zur detaillierten Erzählung ihrer
               Krankengeschichte auffordert, ein, alles mit zu sagen, was ihnen dabei durch den
               Kopf geht, auch wenn siemeinen, es sei unwichtig, oder es gehöre nicht dazu, oder es sei
               unsinnig. Mit besonderem \edtext{Nachdrucke}{\lemma{\textbf{Nachdrucke}}\Aendnote{Nachdruck ES1 }} aber wird von ihnen verlangt, \edtext{daß}{\lemma{\textbf{daß}}\Aendnote{dass E }} sie keinen Gedanken oder Einfall darum von der Mitteilung \edtext{ausschließen}{\lemma{\textbf{ausschließen}}\Aendnote{ausschliessen E }}, weil ihnen diese Mitteilung beschämend oder peinlich ist. Bei den
               Bemühungen, dieses Material an sonst vernachlässigten Einfällen zu sammeln,
               machte nun \so{Freud} die Beobachtungen, die für
               seine ganze Auffassung bestimmend geworden sind. Schon bei der Erzählung\edindex[kw]{Schlagwort Nr. 766} der Krankengeschichte stellen
               sich bei den Kranken Lücken der Erinnerung heraus, sei es, \edtext{daß}{\lemma{\textbf{daß}}\Aendnote{dass E }} tatsächliche Vorgänge vergessen worden, sei es, \edtext{daß}{\lemma{\textbf{daß}}\Aendnote{dass E }} zeitliche Beziehungen verwirrt oder Kausalzusammenhänge zerrissen worden sind, so \edtext{daß}{\lemma{\textbf{daß}}\Aendnote{dass E }} sich unbegreifliche Effekte ergeben. Ohne Amnesie irgend einer Art gibt
               es keine neurotische Krankengeschichte. Drängt man den Erzählenden,
               diese Lücken seines Gedächtnisses durch angestrengte Arbeit der Aufmerksamkeit
                  auszufüllen, so merkt man, \edtext{daß}{\lemma{\textbf{daß}}\Aendnote{dass E }} die \edtext{hiezu}{\lemma{\textbf{hiezu}}\Aendnote{hierzu ES1S2S3S4 }} sich einstellenden Einfälle von ihm mit allen Mitteln der
               Kritik zurückgedrängt werden, bis er endlich das direkte Unbehagen verspürt,
               wenn sich die Erinnerung wirklich eingestellt hat. Aus dieser Erfahrung
               \edtext{schließt}{\lemma{\textbf{schließt}}\Aendnote{schliesst E }}
               \so{Freud}, \edtext{daß}{\lemma{\textbf{daß}}\Aendnote{dass E }} die Amnesien das Ergebnis eines \edtext{Vorganges}{\lemma{\textbf{Vorganges}}\Aendnote{Vorgangs ES1 }} sind, den er \so{Verdrängung}
               \edtext{heißt,}{\lemma{\textbf{heißt,}}\Aendnote{heisst E }} und als dessen Motiv er Unlustgefühle erkennt. Die psychischen Kräfte,
               welche diese Verdrängung herbeigeführt haben, meint er in dem \so{Widerstand}, der sich gegen die Wiederherstellung erhebt, zu
                  verspüren.
        \pend
    


            
        \pstart
        Das Moment des Widerstandes ist eines der Fundamente seiner Theorie geworden.
               Die sonst unter allerlei Vorwänden (wie sie die obige \edtext{Formel}{\lemma{\textbf{Formel}}\Aendnote{Anzahl E }} aufzählt) beseitigten Einfälle betrachtet er aber als Abkömmlinge der
               verdrängten psychischen Gebilde (Gedanken und Regungen\edtext{),}{\lemma{\textbf{),}}\Aendnote{) E }} als Entstellungen derselben infolge des gegen ihre Reproduktion
               bestehenden Widerstandes.
        \pend
    
            
        \pstart
        Je größer \edtext{größer}{\lemma{\textbf{größer}}\Aendnote{grösser E }} der Widerstand, desto ausgiebiger diese Entstellung\edindex[kw]{Schlagwort Nr. 658}. In dieser Beziehung der unbeabsichtigten
               Einfälle zum verdrängten psychischen Material ruht nun ihr Wert für
               die therapeutische Technik. Wenn man ein Verfahren besitzt, welches ermöglicht,
               von den Einfällen aus zu dem Verdrängten, von den Entstellungen zum Entstellten
               zu gelangen, so kann man auch ohne Hypnose das früher \edtext{Unbewußte\edindex[kw]{Schlagwort Nr. 2913}}{\lemma{\textbf{Unbewußte\edindex[kw]{Schlagwort Nr. 2913}}}\Aendnote{unbewusste E }} im Seelenleben dem
               \edtext{Bewußtsein}{\lemma{\textbf{Bewußtsein}}\Aendnote{Bewusstsein E }} zugänglich machen.
        \pend
    


            
        \pstart
        \so{Freud} hat darauf eine \so{Deutungskunst}
                ausgebildet, welcher
               diese Leistung zufällt, die gleichsam aus den Erzen der unbeabsichtigten Einfälle den Metallgehalt an verdrängten Gedanken darstellen soll.
               Objekt dieser Deutungsarbeit sind nicht allein die Einfälle \edtext{des}{\lemma{\textbf{des}}\Aendnote{der S1S2S3S4 }} Kranken, sondern auch seine Träume\edindex[kw]{Schlagwort Nr. 2750}, die den direktesten Zugang zur Kenntnis des \edtext{Unbewußten}{\lemma{\textbf{Unbewußten}}\Aendnote{Unbewussten E }}
               eröffnen, seine
               unbeabsichtigten, wie planlosen Handlungen (Symptomhandlungen) und die Irrungen
               seiner Leistungen im Alltagsleben \edtext{(Versprechen,}{\lemma{\textbf{(Versprechen,}}\Aendnote{Versprechen, E }}
               Vergreifen u. \edtext{dgl.}{\lemma{\textbf{dgl.}}\Aendnote{dergl. E }}). Die Details dieser Deutungs- oder Übersetzungstechnik sind
               von \so{Freud} noch nicht veröffentlicht
                     worden\edtext{}{\Bendnote{\textbf{noch nicht veröffentlicht worden}Zur Psychopathologie des Alltagslebens
                  erscheint 1904.}}. Es sind nach seinen Andeutungen eine Reihe von
               empirisch gewonnenen Regeln, wie aus den Einfällen das \edtext{unbewußte}{\lemma{\textbf{unbewußte}}\Aendnote{unbewusste E }} Material zu konstruieren ist, Anweisungen, wie man es zu verstehen
               habe, wenn die Einfälle des Patienten versagen, und Erfahrungen über die \edtext{wichtigsten}{\lemma{\textbf{wichtigsten}}\Aendnote{richtigsten E }} typischen Widerstände, die sich im Laufe einer solchen Behandlung \edtext{einstellen.}{\lemma{\textbf{einstellen.}}\Aendnote{einstellen E }} Ein umfangreiches Buch über\edtext{ „Traumdeutung “,}{\lemma{\textbf{ „Traumdeutung “,}}\Aendnote{die „Traumdeutung“ E }} 1900 von \so{Freud} publiziert, ist als
               Vorläufer einer solchen Einführung in die Technik anzusehen.
        \pend
    
            
        \pstart
        Man könnte aus diesen Andeutungen über die Technik der psychoanalytischen
               Methode \edtext{schließen}{\lemma{\textbf{schließen}}\Aendnote{schliessen E }}, \edtext{daß}{\lemma{\textbf{daß}}\Aendnote{dass E }} deren Erfinder sich überflüssige Mühe verursacht und Unrecht getan hat,
               das wenig komplizierte hypnotische Verfahren zu verlassen. Aber einerseits
               ist die Technik \edtext{der}{\lemma{\textbf{der}}\Aendnote{der, E }} Psychoanalyse viel leichter auszuüben, wenn man sie einmal erlernt hat,
               als es bei einer Beschreibung den Anschein hat, \edtext{anderseits}{\lemma{\textbf{anderseits}}\Aendnote{andererseits E }} führt kein anderer Weg zum \edtext{Ziele,}{\lemma{\textbf{Ziele,}}\Aendnote{Ziele E }} und darum ist der mühselige Weg noch der kürzeste. Der Hypnose ist
               vorzuwerfen, \edtext{daß}{\lemma{\textbf{daß}}\Aendnote{dass E }} sie den Widerstand verdeckt und dadurch dem Arzt den Einblick in das
               Spiel der psychischen Kräfte verwehrt hat. Sie räumt aber mit dem Widerstande nicht auf, sondern weicht
               ihm nur aus und ergibt \edtext{darum}{\lemma{\textbf{darum}}\Aendnote{dagegen ES1S2S3S4 }} nur unvollständige Auskünfte und nur vorübergehende
                  Erfolge.
        \pend
    


            
        \pstart
        Die Aufgabe, welche die psychoanalytische Methode zu lösen bestrebt ist, \edtext{läßt}{\lemma{\textbf{läßt}}\Aendnote{lässt E }} sich in verschiedenen Formeln ausdrücken, die aber ihrem Wesen nach
               äquivalent sind. Man kann sagen: Aufgabe der Kur sei, die Amnesien
               aufzuheben. Wenn alle Erinnerungslücken ausgefüllt, alle rätselhaften Effekte
               des psychischen Lebens aufgeklärt sind, ist der Fortbestand, ja
               eine Neubildung des Leidens unmöglich gemacht. Man kann die Bedingung
               anders fassen: es seien alle Verdrängungen rückgängig zu machen; der psychische
               Zustand ist dann derselbe, in dem alle Amnesien ausgefüllt sind. Weittragender
               ist eine andere Fassung: es handle sich \edtext{darum,}{\lemma{\textbf{darum,}}\Aendnote{darum E }} das \edtext{Unbewußte}{\lemma{\textbf{Unbewußte}}\Aendnote{Unbewusste E }} dem \edtext{Bewußtsein}{\lemma{\textbf{Bewußtsein}}\Aendnote{Bewusstsein E }} zugänglich zu machen, was durch Überwindung der Widerstände
               geschieht. Man darf aber dabei nicht vergessen, \edtext{daß}{\lemma{\textbf{daß}}\Aendnote{dass E }} ein solcher
               \edtext{Idealzustand}{\lemma{\textbf{Idealzustand}}\Aendnote{Wachzustand E }} auch beim normalen Menschen nicht \edtext{besteht,}{\lemma{\textbf{besteht,}}\Aendnote{besteht E }} und
               \edtext{daß}{\lemma{\textbf{daß}}\Aendnote{dass E }} man nur selten in die Lage kommen kann, die Behandlung annähernd so weit
               zu treiben. So wie Gesundheit und Krankheit nicht prinzipiell geschieden,
               sondern nur durch eine praktisch bestimmbare Summationsgrenze gesondert sind, so
               wird man sich auch nie etwas anderes zum Ziel der Behandlung setzen als die
               praktische Genesung des Kranken, die Herstellung seiner Leistungs-  und \edtext{Genußfähigkeit}{\lemma{\textbf{Genußfähigkeit}}\Aendnote{Genussfähigkeit E }}. Bei unvollständiger Kur oder unvollkommenem \edtext{Erfolge}{\lemma{\textbf{Erfolge}}\Aendnote{Erfolg E }} derselben erreicht man vor allem eine bedeutende Hebung des psychischen
               Allgemeinzustandes, während die Symptome, aber mit geminderter Bedeutung für den \edtext{Kranken,}{\lemma{\textbf{Kranken,}}\Aendnote{Kranken E }} fortbestehen können, ohne ihn zu einem Kranken zu stempeln.
        \pend
    
            
        \pstart
        Das therapeutische Verfahren \edtext{bleibt,}{\lemma{\textbf{bleibt,}}\Aendnote{bleibt E }} von geringen Modifikationen
               \edtext{abgesehen,}{\lemma{\textbf{abgesehen,}}\Aendnote{abgesehen E }} das nämliche für alle Symptombilder der vielgestaltigenHysterie\edindex[kw]{Schlagwort Nr. 1224} und ebenso für alle Ausbildungen der Zwangsneurose. Von einer unbeschränkten
               Anwendbarkeit desselben ist aber keine Rede. Die Natur der \edtext{psychoanalytischen}{\lemma{\textbf{psychoanalytischen}}\Aendnote{psycho-analytischen E }} Methode schafft Indikationen und Gegenanzeigen sowohl von \edtext{seiten}{\lemma{\textbf{seiten}}\Aendnote{Seiten E }} der zu behandelnden \edtext{Personen}{\lemma{\textbf{Personen}}\Aendnote{Personen, ES1S2S3S4 }} als auch mit Rücksicht auf das Krankheitsbild. Am günstigsten für die
               Psychoanalyse sind die chronischen Fälle von Psychoneurosen mit wenig
               stürmischen oder gefahrdrohenden Symptomen, also zunächst alle Arten der
               Zwangsneurose, Zwangsdenken und Zwangshandeln, und Fälle von Hysterie, in
               denen Phobien und Abulien \edtext{}{\Bendnote{\textbf{Abulien}Abulie, griech. krankhafte
                  Willensschwäche}} die Hauptrolle spielen, weiterhin aber auch alle
               somatischen Ausprägungen der Hysterie, \edtext{insoferne}{\lemma{\textbf{insoferne}}\Aendnote{insofern E }}
               \edtext{nicht,}{\lemma{\textbf{nicht,}}\Aendnote{nicht E }} wie bei der \edtext{Anorexie,}{\lemma{\textbf{Anorexie,}}\Aendnote{Anorexie E }} rasche Beseitigung der Symptome zur Hauptaufgabe des Arztes wird. Bei
               akuten Fällen von Hysterie wird man den Eintritt eines ruhigeren Stadiums
               abzuwarten haben; in allen Fällen, bei denen die nervöse Erschöpfung obenan
               steht, wird man ein Verfahren vermeiden, welches selbst Anstrengung erfordert,
               nur langsame Fortschritte \edtext{zeitigt}{\lemma{\textbf{zeitigt}}\Aendnote{zeitigt, E }} und auf die Fortdauer der Symptome eine Zeitlang keine Rücksicht nehmen
                  kann.
        \pend
    


            
        \pstart
        An die Person, die man mit Vorteil der Psychoanalyse unterziehen
               soll, sind mehrfache Forderungen zu stellen. Sie \edtext{muß}{\lemma{\textbf{muß}}\Aendnote{muss E }} erstens eines psychischen Normalzustandes fähig sein; in Zeiten der Verworrenheit oder
               melancholischer Depression ist auch bei einer Hysterie nichts auszurichten. Man
               darf ferner ein gewisses
               \edtext{Maß}{\lemma{\textbf{Maß}}\Aendnote{Maass E }} natürlicher Intelligenz und ethischer Entwicklung fordern; bei wertlosen
               Personen \edtext{läßt}{\lemma{\textbf{läßt}}\Aendnote{lässt E }} den Arzt bald das Interesse im Stiche, welches ihn zur Vertiefung in das
               Seelenlebens des Kranken befähigt. Ausgeprägte Charakterverbildungen, Züge
               von wirklich degenerativer Konstitution \edtext{äußern}{\lemma{\textbf{äußern}}\Aendnote{äussern E }} sich bei der Kur als Quelle von kaum zu überwindenden Widerständen.
               Insoweit setzt überhaupt die Konstitution eine Grenze für die Heilbarkeit
               durch Psychotherapie. Auch eine Altersstufe in der Nähe des fünften
               Dezenniums schafft ungünstige Bedingungen für die Psychoanalyse. Die
               Masse des psychischen Materials ist dann nicht mehr zu bewältigen, die zur
               Herstellung erforderliche Zeit wird zu \edtext{lang,}{\lemma{\textbf{lang,}}\Aendnote{lang E }} und die Fähigkeit, psychische Vorgänge rückgängig zu machen, beginnt zu
                  erlahmen.
        \pend
    

            
            
            
        \pstart
        Trotz aller dieser Einschränkungen ist die Anzahl der für die Psychoanalyse
               geeigneten Personen eine\edtext{außerordentlich große}{\lemma{\textbf{außerordentlich große}}\Aendnote{ausserordentlich grosse E außerordentlich große, S1S2S3S4 }} und die Erweiterung unseres therapeutischen Könnens durch dieses
               Verfahren nach den Behauptungen \edtext{\so{Freuds}}{\lemma{\textbf{\so{Freuds}}}\Aendnote{\so{Freud}’s E }} eine sehr beträchtliche. \so{Freud}
               beansprucht lange Zeiträume, \edtext{ein halbes}{\lemma{\textbf{ein halbes}}\Aendnote{½ ES1S2S3S4 }} bis \edtext{drei}{\lemma{\textbf{drei}}\Aendnote{3 ES1S2S3S4 }} Jahre für eine wirksame Behandlung; er gibt aber die Auskunft, \edtext{daß}{\lemma{\textbf{daß}}\Aendnote{dass E }} er bisher infolge verschiedener leicht zu erratender Umstände meist nur
               in die Lage gekommen ist, seine Behandlung an sehr schweren Fällen zu erproben,
               Personen mit vieljähriger Krankheitsdauer und völliger Leistungsunfähigkeit, \edtext{die,}{\lemma{\textbf{die,}}\Aendnote{die E }} durch alle Behandlungen getäuscht, gleichsam eine letzte Zuflucht bei
               seinem neuen und viel abgezweifelten Verfahren gesucht \edtext{haben.}{\lemma{\textbf{haben.}}\Aendnote{haben S4 }} In Fällen leichterer Erkrankung dürfte sich die Behandlungsdauer
               sehr verkürzen und ein \edtext{außerordentlicher}{\lemma{\textbf{außerordentlicher}}\Aendnote{ausserordentlicher E außerodentlicher S4 }} Gewinn an Vorbeugung für die Zukunft erzielen \edtext{lassen.“}{\lemma{\textbf{lassen.“}}\Aendnote{lassen“. E }}
               
            
        \pend
    
         
        \endnumbering
        
        \section*{Herausgebereingriffe}
        \doendnotes{C}
  
        \section*{Kritischer Apparat}
        \doendnotes{A}
        
        \section*{Stellenkommentar}
        \doendnotes{B}
        
        \printindex[person]
        \printindex[kw]
        


        \end{document}
    